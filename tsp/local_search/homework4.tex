\documentclass[11pt]{article}

    \usepackage[breakable]{tcolorbox}
    \usepackage{parskip} % Stop auto-indenting (to mimic markdown behaviour)
    
    \usepackage{iftex}
    \ifPDFTeX
    	\usepackage[T1]{fontenc}
    	\usepackage{mathpazo}
    \else
    	\usepackage{fontspec}
    \fi

    % Basic figure setup, for now with no caption control since it's done
    % automatically by Pandoc (which extracts ![](path) syntax from Markdown).
    \usepackage{graphicx}
    % Maintain compatibility with old templates. Remove in nbconvert 6.0
    \let\Oldincludegraphics\includegraphics
    % Ensure that by default, figures have no caption (until we provide a
    % proper Figure object with a Caption API and a way to capture that
    % in the conversion process - todo).
    \usepackage{caption}
    \DeclareCaptionFormat{nocaption}{}
    \captionsetup{format=nocaption,aboveskip=0pt,belowskip=0pt}

    \usepackage[Export]{adjustbox} % Used to constrain images to a maximum size
    \adjustboxset{max size={0.9\linewidth}{0.9\paperheight}}
    \usepackage{float}
    \floatplacement{figure}{H} % forces figures to be placed at the correct location
    \usepackage{xcolor} % Allow colors to be defined
    \usepackage{enumerate} % Needed for markdown enumerations to work
    \usepackage{geometry} % Used to adjust the document margins
    \usepackage{amsmath} % Equations
    \usepackage{amssymb} % Equations
    \usepackage{textcomp} % defines textquotesingle
    % Hack from http://tex.stackexchange.com/a/47451/13684:
    \AtBeginDocument{%
        \def\PYZsq{\textquotesingle}% Upright quotes in Pygmentized code
    }
    \usepackage{upquote} % Upright quotes for verbatim code
    \usepackage{eurosym} % defines \euro
    \usepackage[mathletters]{ucs} % Extended unicode (utf-8) support
    \usepackage{fancyvrb} % verbatim replacement that allows latex
    \usepackage{grffile} % extends the file name processing of package graphics 
                         % to support a larger range
    \makeatletter % fix for grffile with XeLaTeX
    \def\Gread@@xetex#1{%
      \IfFileExists{"\Gin@base".bb}%
      {\Gread@eps{\Gin@base.bb}}%
      {\Gread@@xetex@aux#1}%
    }
    \makeatother

    % The hyperref package gives us a pdf with properly built
    % internal navigation ('pdf bookmarks' for the table of contents,
    % internal cross-reference links, web links for URLs, etc.)
    \usepackage{hyperref}
    % The default LaTeX title has an obnoxious amount of whitespace. By default,
    % titling removes some of it. It also provides customization options.
    \usepackage{titling}
    \usepackage{longtable} % longtable support required by pandoc >1.10
    \usepackage{booktabs}  % table support for pandoc > 1.12.2
    \usepackage[inline]{enumitem} % IRkernel/repr support (it uses the enumerate* environment)
    \usepackage[normalem]{ulem} % ulem is needed to support strikethroughs (\sout)
                                % normalem makes italics be italics, not underlines
    \usepackage{mathrsfs}

    % Colors for the hyperref package
    \definecolor{urlcolor}{rgb}{0,.145,.698}
    \definecolor{linkcolor}{rgb}{.71,0.21,0.01}
    \definecolor{citecolor}{rgb}{.12,.54,.11}

    % ANSI colors
    \definecolor{ansi-black}{HTML}{3E424D}
    \definecolor{ansi-black-intense}{HTML}{282C36}
    \definecolor{ansi-red}{HTML}{E75C58}
    \definecolor{ansi-red-intense}{HTML}{B22B31}
    \definecolor{ansi-green}{HTML}{00A250}
    \definecolor{ansi-green-intense}{HTML}{007427}
    \definecolor{ansi-yellow}{HTML}{DDB62B}
    \definecolor{ansi-yellow-intense}{HTML}{B27D12}
    \definecolor{ansi-blue}{HTML}{208FFB}
    \definecolor{ansi-blue-intense}{HTML}{0065CA}
    \definecolor{ansi-magenta}{HTML}{D160C4}
    \definecolor{ansi-magenta-intense}{HTML}{A03196}
    \definecolor{ansi-cyan}{HTML}{60C6C8}
    \definecolor{ansi-cyan-intense}{HTML}{258F8F}
    \definecolor{ansi-white}{HTML}{C5C1B4}
    \definecolor{ansi-white-intense}{HTML}{A1A6B2}
    \definecolor{ansi-default-inverse-fg}{HTML}{FFFFFF}
    \definecolor{ansi-default-inverse-bg}{HTML}{000000}

    % commands and environments needed by pandoc snippets
    % extracted from the output of `pandoc -s`
    \providecommand{\tightlist}{%
      \setlength{\itemsep}{0pt}\setlength{\parskip}{0pt}}
    \DefineVerbatimEnvironment{Highlighting}{Verbatim}{commandchars=\\\{\}}
    % Add ',fontsize=\small' for more characters per line
    \newenvironment{Shaded}{}{}
    \newcommand{\KeywordTok}[1]{\textcolor[rgb]{0.00,0.44,0.13}{\textbf{{#1}}}}
    \newcommand{\DataTypeTok}[1]{\textcolor[rgb]{0.56,0.13,0.00}{{#1}}}
    \newcommand{\DecValTok}[1]{\textcolor[rgb]{0.25,0.63,0.44}{{#1}}}
    \newcommand{\BaseNTok}[1]{\textcolor[rgb]{0.25,0.63,0.44}{{#1}}}
    \newcommand{\FloatTok}[1]{\textcolor[rgb]{0.25,0.63,0.44}{{#1}}}
    \newcommand{\CharTok}[1]{\textcolor[rgb]{0.25,0.44,0.63}{{#1}}}
    \newcommand{\StringTok}[1]{\textcolor[rgb]{0.25,0.44,0.63}{{#1}}}
    \newcommand{\CommentTok}[1]{\textcolor[rgb]{0.38,0.63,0.69}{\textit{{#1}}}}
    \newcommand{\OtherTok}[1]{\textcolor[rgb]{0.00,0.44,0.13}{{#1}}}
    \newcommand{\AlertTok}[1]{\textcolor[rgb]{1.00,0.00,0.00}{\textbf{{#1}}}}
    \newcommand{\FunctionTok}[1]{\textcolor[rgb]{0.02,0.16,0.49}{{#1}}}
    \newcommand{\RegionMarkerTok}[1]{{#1}}
    \newcommand{\ErrorTok}[1]{\textcolor[rgb]{1.00,0.00,0.00}{\textbf{{#1}}}}
    \newcommand{\NormalTok}[1]{{#1}}
    
    % Additional commands for more recent versions of Pandoc
    \newcommand{\ConstantTok}[1]{\textcolor[rgb]{0.53,0.00,0.00}{{#1}}}
    \newcommand{\SpecialCharTok}[1]{\textcolor[rgb]{0.25,0.44,0.63}{{#1}}}
    \newcommand{\VerbatimStringTok}[1]{\textcolor[rgb]{0.25,0.44,0.63}{{#1}}}
    \newcommand{\SpecialStringTok}[1]{\textcolor[rgb]{0.73,0.40,0.53}{{#1}}}
    \newcommand{\ImportTok}[1]{{#1}}
    \newcommand{\DocumentationTok}[1]{\textcolor[rgb]{0.73,0.13,0.13}{\textit{{#1}}}}
    \newcommand{\AnnotationTok}[1]{\textcolor[rgb]{0.38,0.63,0.69}{\textbf{\textit{{#1}}}}}
    \newcommand{\CommentVarTok}[1]{\textcolor[rgb]{0.38,0.63,0.69}{\textbf{\textit{{#1}}}}}
    \newcommand{\VariableTok}[1]{\textcolor[rgb]{0.10,0.09,0.49}{{#1}}}
    \newcommand{\ControlFlowTok}[1]{\textcolor[rgb]{0.00,0.44,0.13}{\textbf{{#1}}}}
    \newcommand{\OperatorTok}[1]{\textcolor[rgb]{0.40,0.40,0.40}{{#1}}}
    \newcommand{\BuiltInTok}[1]{{#1}}
    \newcommand{\ExtensionTok}[1]{{#1}}
    \newcommand{\PreprocessorTok}[1]{\textcolor[rgb]{0.74,0.48,0.00}{{#1}}}
    \newcommand{\AttributeTok}[1]{\textcolor[rgb]{0.49,0.56,0.16}{{#1}}}
    \newcommand{\InformationTok}[1]{\textcolor[rgb]{0.38,0.63,0.69}{\textbf{\textit{{#1}}}}}
    \newcommand{\WarningTok}[1]{\textcolor[rgb]{0.38,0.63,0.69}{\textbf{\textit{{#1}}}}}
    
    % Define a nice break command that doesn't care if a line doesn't already
    % exist.
    \def\br{\hspace*{\fill} \\* }
    % Math Jax compatibility definitions
    \def\gt{>}
    \def\lt{<}
    \let\Oldtex\TeX
    \let\Oldlatex\LaTeX
    \renewcommand{\TeX}{\textrm{\Oldtex}}
    \renewcommand{\LaTeX}{\textrm{\Oldlatex}}
    % Document parameters
    % Document title
    \title{
        Travelling Salesman Problem \\
        \large Local search heuristic
    }
    \author{Ernesto Ortiz}
    
% Pygments definitions
\makeatletter
\def\PY@reset{\let\PY@it=\relax \let\PY@bf=\relax%
    \let\PY@ul=\relax \let\PY@tc=\relax%
    \let\PY@bc=\relax \let\PY@ff=\relax}
\def\PY@tok#1{\csname PY@tok@#1\endcsname}
\def\PY@toks#1+{\ifx\relax#1\empty\else%
    \PY@tok{#1}\expandafter\PY@toks\fi}
\def\PY@do#1{\PY@bc{\PY@tc{\PY@ul{%
    \PY@it{\PY@bf{\PY@ff{#1}}}}}}}
\def\PY#1#2{\PY@reset\PY@toks#1+\relax+\PY@do{#2}}

\expandafter\def\csname PY@tok@w\endcsname{\def\PY@tc##1{\textcolor[rgb]{0.73,0.73,0.73}{##1}}}
\expandafter\def\csname PY@tok@c\endcsname{\let\PY@it=\textit\def\PY@tc##1{\textcolor[rgb]{0.25,0.50,0.50}{##1}}}
\expandafter\def\csname PY@tok@cp\endcsname{\def\PY@tc##1{\textcolor[rgb]{0.74,0.48,0.00}{##1}}}
\expandafter\def\csname PY@tok@k\endcsname{\let\PY@bf=\textbf\def\PY@tc##1{\textcolor[rgb]{0.00,0.50,0.00}{##1}}}
\expandafter\def\csname PY@tok@kp\endcsname{\def\PY@tc##1{\textcolor[rgb]{0.00,0.50,0.00}{##1}}}
\expandafter\def\csname PY@tok@kt\endcsname{\def\PY@tc##1{\textcolor[rgb]{0.69,0.00,0.25}{##1}}}
\expandafter\def\csname PY@tok@o\endcsname{\def\PY@tc##1{\textcolor[rgb]{0.40,0.40,0.40}{##1}}}
\expandafter\def\csname PY@tok@ow\endcsname{\let\PY@bf=\textbf\def\PY@tc##1{\textcolor[rgb]{0.67,0.13,1.00}{##1}}}
\expandafter\def\csname PY@tok@nb\endcsname{\def\PY@tc##1{\textcolor[rgb]{0.00,0.50,0.00}{##1}}}
\expandafter\def\csname PY@tok@nf\endcsname{\def\PY@tc##1{\textcolor[rgb]{0.00,0.00,1.00}{##1}}}
\expandafter\def\csname PY@tok@nc\endcsname{\let\PY@bf=\textbf\def\PY@tc##1{\textcolor[rgb]{0.00,0.00,1.00}{##1}}}
\expandafter\def\csname PY@tok@nn\endcsname{\let\PY@bf=\textbf\def\PY@tc##1{\textcolor[rgb]{0.00,0.00,1.00}{##1}}}
\expandafter\def\csname PY@tok@ne\endcsname{\let\PY@bf=\textbf\def\PY@tc##1{\textcolor[rgb]{0.82,0.25,0.23}{##1}}}
\expandafter\def\csname PY@tok@nv\endcsname{\def\PY@tc##1{\textcolor[rgb]{0.10,0.09,0.49}{##1}}}
\expandafter\def\csname PY@tok@no\endcsname{\def\PY@tc##1{\textcolor[rgb]{0.53,0.00,0.00}{##1}}}
\expandafter\def\csname PY@tok@nl\endcsname{\def\PY@tc##1{\textcolor[rgb]{0.63,0.63,0.00}{##1}}}
\expandafter\def\csname PY@tok@ni\endcsname{\let\PY@bf=\textbf\def\PY@tc##1{\textcolor[rgb]{0.60,0.60,0.60}{##1}}}
\expandafter\def\csname PY@tok@na\endcsname{\def\PY@tc##1{\textcolor[rgb]{0.49,0.56,0.16}{##1}}}
\expandafter\def\csname PY@tok@nt\endcsname{\let\PY@bf=\textbf\def\PY@tc##1{\textcolor[rgb]{0.00,0.50,0.00}{##1}}}
\expandafter\def\csname PY@tok@nd\endcsname{\def\PY@tc##1{\textcolor[rgb]{0.67,0.13,1.00}{##1}}}
\expandafter\def\csname PY@tok@s\endcsname{\def\PY@tc##1{\textcolor[rgb]{0.73,0.13,0.13}{##1}}}
\expandafter\def\csname PY@tok@sd\endcsname{\let\PY@it=\textit\def\PY@tc##1{\textcolor[rgb]{0.73,0.13,0.13}{##1}}}
\expandafter\def\csname PY@tok@si\endcsname{\let\PY@bf=\textbf\def\PY@tc##1{\textcolor[rgb]{0.73,0.40,0.53}{##1}}}
\expandafter\def\csname PY@tok@se\endcsname{\let\PY@bf=\textbf\def\PY@tc##1{\textcolor[rgb]{0.73,0.40,0.13}{##1}}}
\expandafter\def\csname PY@tok@sr\endcsname{\def\PY@tc##1{\textcolor[rgb]{0.73,0.40,0.53}{##1}}}
\expandafter\def\csname PY@tok@ss\endcsname{\def\PY@tc##1{\textcolor[rgb]{0.10,0.09,0.49}{##1}}}
\expandafter\def\csname PY@tok@sx\endcsname{\def\PY@tc##1{\textcolor[rgb]{0.00,0.50,0.00}{##1}}}
\expandafter\def\csname PY@tok@m\endcsname{\def\PY@tc##1{\textcolor[rgb]{0.40,0.40,0.40}{##1}}}
\expandafter\def\csname PY@tok@gh\endcsname{\let\PY@bf=\textbf\def\PY@tc##1{\textcolor[rgb]{0.00,0.00,0.50}{##1}}}
\expandafter\def\csname PY@tok@gu\endcsname{\let\PY@bf=\textbf\def\PY@tc##1{\textcolor[rgb]{0.50,0.00,0.50}{##1}}}
\expandafter\def\csname PY@tok@gd\endcsname{\def\PY@tc##1{\textcolor[rgb]{0.63,0.00,0.00}{##1}}}
\expandafter\def\csname PY@tok@gi\endcsname{\def\PY@tc##1{\textcolor[rgb]{0.00,0.63,0.00}{##1}}}
\expandafter\def\csname PY@tok@gr\endcsname{\def\PY@tc##1{\textcolor[rgb]{1.00,0.00,0.00}{##1}}}
\expandafter\def\csname PY@tok@ge\endcsname{\let\PY@it=\textit}
\expandafter\def\csname PY@tok@gs\endcsname{\let\PY@bf=\textbf}
\expandafter\def\csname PY@tok@gp\endcsname{\let\PY@bf=\textbf\def\PY@tc##1{\textcolor[rgb]{0.00,0.00,0.50}{##1}}}
\expandafter\def\csname PY@tok@go\endcsname{\def\PY@tc##1{\textcolor[rgb]{0.53,0.53,0.53}{##1}}}
\expandafter\def\csname PY@tok@gt\endcsname{\def\PY@tc##1{\textcolor[rgb]{0.00,0.27,0.87}{##1}}}
\expandafter\def\csname PY@tok@err\endcsname{\def\PY@bc##1{\setlength{\fboxsep}{0pt}\fcolorbox[rgb]{1.00,0.00,0.00}{1,1,1}{\strut ##1}}}
\expandafter\def\csname PY@tok@kc\endcsname{\let\PY@bf=\textbf\def\PY@tc##1{\textcolor[rgb]{0.00,0.50,0.00}{##1}}}
\expandafter\def\csname PY@tok@kd\endcsname{\let\PY@bf=\textbf\def\PY@tc##1{\textcolor[rgb]{0.00,0.50,0.00}{##1}}}
\expandafter\def\csname PY@tok@kn\endcsname{\let\PY@bf=\textbf\def\PY@tc##1{\textcolor[rgb]{0.00,0.50,0.00}{##1}}}
\expandafter\def\csname PY@tok@kr\endcsname{\let\PY@bf=\textbf\def\PY@tc##1{\textcolor[rgb]{0.00,0.50,0.00}{##1}}}
\expandafter\def\csname PY@tok@bp\endcsname{\def\PY@tc##1{\textcolor[rgb]{0.00,0.50,0.00}{##1}}}
\expandafter\def\csname PY@tok@fm\endcsname{\def\PY@tc##1{\textcolor[rgb]{0.00,0.00,1.00}{##1}}}
\expandafter\def\csname PY@tok@vc\endcsname{\def\PY@tc##1{\textcolor[rgb]{0.10,0.09,0.49}{##1}}}
\expandafter\def\csname PY@tok@vg\endcsname{\def\PY@tc##1{\textcolor[rgb]{0.10,0.09,0.49}{##1}}}
\expandafter\def\csname PY@tok@vi\endcsname{\def\PY@tc##1{\textcolor[rgb]{0.10,0.09,0.49}{##1}}}
\expandafter\def\csname PY@tok@vm\endcsname{\def\PY@tc##1{\textcolor[rgb]{0.10,0.09,0.49}{##1}}}
\expandafter\def\csname PY@tok@sa\endcsname{\def\PY@tc##1{\textcolor[rgb]{0.73,0.13,0.13}{##1}}}
\expandafter\def\csname PY@tok@sb\endcsname{\def\PY@tc##1{\textcolor[rgb]{0.73,0.13,0.13}{##1}}}
\expandafter\def\csname PY@tok@sc\endcsname{\def\PY@tc##1{\textcolor[rgb]{0.73,0.13,0.13}{##1}}}
\expandafter\def\csname PY@tok@dl\endcsname{\def\PY@tc##1{\textcolor[rgb]{0.73,0.13,0.13}{##1}}}
\expandafter\def\csname PY@tok@s2\endcsname{\def\PY@tc##1{\textcolor[rgb]{0.73,0.13,0.13}{##1}}}
\expandafter\def\csname PY@tok@sh\endcsname{\def\PY@tc##1{\textcolor[rgb]{0.73,0.13,0.13}{##1}}}
\expandafter\def\csname PY@tok@s1\endcsname{\def\PY@tc##1{\textcolor[rgb]{0.73,0.13,0.13}{##1}}}
\expandafter\def\csname PY@tok@mb\endcsname{\def\PY@tc##1{\textcolor[rgb]{0.40,0.40,0.40}{##1}}}
\expandafter\def\csname PY@tok@mf\endcsname{\def\PY@tc##1{\textcolor[rgb]{0.40,0.40,0.40}{##1}}}
\expandafter\def\csname PY@tok@mh\endcsname{\def\PY@tc##1{\textcolor[rgb]{0.40,0.40,0.40}{##1}}}
\expandafter\def\csname PY@tok@mi\endcsname{\def\PY@tc##1{\textcolor[rgb]{0.40,0.40,0.40}{##1}}}
\expandafter\def\csname PY@tok@il\endcsname{\def\PY@tc##1{\textcolor[rgb]{0.40,0.40,0.40}{##1}}}
\expandafter\def\csname PY@tok@mo\endcsname{\def\PY@tc##1{\textcolor[rgb]{0.40,0.40,0.40}{##1}}}
\expandafter\def\csname PY@tok@ch\endcsname{\let\PY@it=\textit\def\PY@tc##1{\textcolor[rgb]{0.25,0.50,0.50}{##1}}}
\expandafter\def\csname PY@tok@cm\endcsname{\let\PY@it=\textit\def\PY@tc##1{\textcolor[rgb]{0.25,0.50,0.50}{##1}}}
\expandafter\def\csname PY@tok@cpf\endcsname{\let\PY@it=\textit\def\PY@tc##1{\textcolor[rgb]{0.25,0.50,0.50}{##1}}}
\expandafter\def\csname PY@tok@c1\endcsname{\let\PY@it=\textit\def\PY@tc##1{\textcolor[rgb]{0.25,0.50,0.50}{##1}}}
\expandafter\def\csname PY@tok@cs\endcsname{\let\PY@it=\textit\def\PY@tc##1{\textcolor[rgb]{0.25,0.50,0.50}{##1}}}

\def\PYZbs{\char`\\}
\def\PYZus{\char`\_}
\def\PYZob{\char`\{}
\def\PYZcb{\char`\}}
\def\PYZca{\char`\^}
\def\PYZam{\char`\&}
\def\PYZlt{\char`\<}
\def\PYZgt{\char`\>}
\def\PYZsh{\char`\#}
\def\PYZpc{\char`\%}
\def\PYZdl{\char`\$}
\def\PYZhy{\char`\-}
\def\PYZsq{\char`\'}
\def\PYZdq{\char`\"}
\def\PYZti{\char`\~}
% for compatibility with earlier versions
\def\PYZat{@}
\def\PYZlb{[}
\def\PYZrb{]}
\makeatother


    % For linebreaks inside Verbatim environment from package fancyvrb. 
    \makeatletter
        \newbox\Wrappedcontinuationbox 
        \newbox\Wrappedvisiblespacebox 
        \newcommand*\Wrappedvisiblespace {\textcolor{red}{\textvisiblespace}} 
        \newcommand*\Wrappedcontinuationsymbol {\textcolor{red}{\llap{\tiny$\m@th\hookrightarrow$}}} 
        \newcommand*\Wrappedcontinuationindent {3ex } 
        \newcommand*\Wrappedafterbreak {\kern\Wrappedcontinuationindent\copy\Wrappedcontinuationbox} 
        % Take advantage of the already applied Pygments mark-up to insert 
        % potential linebreaks for TeX processing. 
        %        {, <, #, %, $, ' and ": go to next line. 
        %        _, }, ^, &, >, - and ~: stay at end of broken line. 
        % Use of \textquotesingle for straight quote. 
        \newcommand*\Wrappedbreaksatspecials {% 
            \def\PYGZus{\discretionary{\char`\_}{\Wrappedafterbreak}{\char`\_}}% 
            \def\PYGZob{\discretionary{}{\Wrappedafterbreak\char`\{}{\char`\{}}% 
            \def\PYGZcb{\discretionary{\char`\}}{\Wrappedafterbreak}{\char`\}}}% 
            \def\PYGZca{\discretionary{\char`\^}{\Wrappedafterbreak}{\char`\^}}% 
            \def\PYGZam{\discretionary{\char`\&}{\Wrappedafterbreak}{\char`\&}}% 
            \def\PYGZlt{\discretionary{}{\Wrappedafterbreak\char`\<}{\char`\<}}% 
            \def\PYGZgt{\discretionary{\char`\>}{\Wrappedafterbreak}{\char`\>}}% 
            \def\PYGZsh{\discretionary{}{\Wrappedafterbreak\char`\#}{\char`\#}}% 
            \def\PYGZpc{\discretionary{}{\Wrappedafterbreak\char`\%}{\char`\%}}% 
            \def\PYGZdl{\discretionary{}{\Wrappedafterbreak\char`\$}{\char`\$}}% 
            \def\PYGZhy{\discretionary{\char`\-}{\Wrappedafterbreak}{\char`\-}}% 
            \def\PYGZsq{\discretionary{}{\Wrappedafterbreak\textquotesingle}{\textquotesingle}}% 
            \def\PYGZdq{\discretionary{}{\Wrappedafterbreak\char`\"}{\char`\"}}% 
            \def\PYGZti{\discretionary{\char`\~}{\Wrappedafterbreak}{\char`\~}}% 
        } 
        % Some characters . , ; ? ! / are not pygmentized. 
        % This macro makes them "active" and they will insert potential linebreaks 
        \newcommand*\Wrappedbreaksatpunct {% 
            \lccode`\~`\.\lowercase{\def~}{\discretionary{\hbox{\char`\.}}{\Wrappedafterbreak}{\hbox{\char`\.}}}% 
            \lccode`\~`\,\lowercase{\def~}{\discretionary{\hbox{\char`\,}}{\Wrappedafterbreak}{\hbox{\char`\,}}}% 
            \lccode`\~`\;\lowercase{\def~}{\discretionary{\hbox{\char`\;}}{\Wrappedafterbreak}{\hbox{\char`\;}}}% 
            \lccode`\~`\:\lowercase{\def~}{\discretionary{\hbox{\char`\:}}{\Wrappedafterbreak}{\hbox{\char`\:}}}% 
            \lccode`\~`\?\lowercase{\def~}{\discretionary{\hbox{\char`\?}}{\Wrappedafterbreak}{\hbox{\char`\?}}}% 
            \lccode`\~`\!\lowercase{\def~}{\discretionary{\hbox{\char`\!}}{\Wrappedafterbreak}{\hbox{\char`\!}}}% 
            \lccode`\~`\/\lowercase{\def~}{\discretionary{\hbox{\char`\/}}{\Wrappedafterbreak}{\hbox{\char`\/}}}% 
            \catcode`\.\active
            \catcode`\,\active 
            \catcode`\;\active
            \catcode`\:\active
            \catcode`\?\active
            \catcode`\!\active
            \catcode`\/\active 
            \lccode`\~`\~ 	
        }
    \makeatother

    \let\OriginalVerbatim=\Verbatim
    \makeatletter
    \renewcommand{\Verbatim}[1][1]{%
        %\parskip\z@skip
        \sbox\Wrappedcontinuationbox {\Wrappedcontinuationsymbol}%
        \sbox\Wrappedvisiblespacebox {\FV@SetupFont\Wrappedvisiblespace}%
        \def\FancyVerbFormatLine ##1{\hsize\linewidth
            \vtop{\raggedright\hyphenpenalty\z@\exhyphenpenalty\z@
                \doublehyphendemerits\z@\finalhyphendemerits\z@
                \strut ##1\strut}%
        }%
        % If the linebreak is at a space, the latter will be displayed as visible
        % space at end of first line, and a continuation symbol starts next line.
        % Stretch/shrink are however usually zero for typewriter font.
        \def\FV@Space {%
            \nobreak\hskip\z@ plus\fontdimen3\font minus\fontdimen4\font
            \discretionary{\copy\Wrappedvisiblespacebox}{\Wrappedafterbreak}
            {\kern\fontdimen2\font}%
        }%
        
        % Allow breaks at special characters using \PYG... macros.
        \Wrappedbreaksatspecials
        % Breaks at punctuation characters . , ; ? ! and / need catcode=\active 	
        \OriginalVerbatim[#1,codes*=\Wrappedbreaksatpunct]%
    }
    \makeatother

    % Exact colors from NB
    \definecolor{incolor}{HTML}{303F9F}
    \definecolor{outcolor}{HTML}{D84315}
    \definecolor{cellborder}{HTML}{CFCFCF}
    \definecolor{cellbackground}{HTML}{F7F7F7}
    
    % prompt
    \makeatletter
    \newcommand{\boxspacing}{\kern\kvtcb@left@rule\kern\kvtcb@boxsep}
    \makeatother
    \newcommand{\prompt}[4]{
        \ttfamily\llap{{\color{#2}[#3]:\hspace{3pt}#4}}\vspace{-\baselineskip}
    }
    

    
    % Prevent overflowing lines due to hard-to-break entities
    \sloppy 
    % Setup hyperref package
    \hypersetup{
      breaklinks=true,  % so long urls are correctly broken across lines
      colorlinks=true,
      urlcolor=urlcolor,
      linkcolor=linkcolor,
      citecolor=citecolor,
      }
    % Slightly bigger margins than the latex defaults
    
    \geometry{verbose,tmargin=1in,bmargin=1in,lmargin=1in,rmargin=1in}
    
    

\begin{document}
    
    \maketitle
    
    

    
    \hypertarget{data}{%
\section{Data}\label{data}}

\begin{itemize}
\item
  \(C=\{1,2,3,4,5,6,7,8\}\) \(\leftarrow\) set of cities to visit.
\item
  \(n = 8\) \(\leftarrow\) number of cities.
\item
  \(T = (1,2,3,4,5,6,7,8)\) \(\leftarrow\) current tour.
\end{itemize}

    \begin{tcolorbox}[breakable, size=fbox, boxrule=1pt, pad at break*=1mm,colback=cellbackground, colframe=cellborder]
\prompt{In}{incolor}{1}{\boxspacing}
\begin{Verbatim}[commandchars=\\\{\}]
\PY{n}{tour} \PY{o}{=} \PY{p}{[}\PY{l+m+mi}{1}\PY{p}{,} \PY{l+m+mi}{2}\PY{p}{,} \PY{l+m+mi}{3}\PY{p}{,} \PY{l+m+mi}{4}\PY{p}{,} \PY{l+m+mi}{5}\PY{p}{,} \PY{l+m+mi}{6}\PY{p}{,} \PY{l+m+mi}{7}\PY{p}{,} \PY{l+m+mi}{8}\PY{p}{]}
\end{Verbatim}
\end{tcolorbox}

    \begin{itemize}
\tightlist
\item
  Coordinates of each city:
\end{itemize}

    \begin{tcolorbox}[breakable, size=fbox, boxrule=1pt, pad at break*=1mm,colback=cellbackground, colframe=cellborder]
\prompt{In}{incolor}{2}{\boxspacing}
\begin{Verbatim}[commandchars=\\\{\}]
\PY{n}{coords} \PY{o}{=} \PY{p}{(}
    \PY{c+c1}{\PYZsh{} x   y    city}
    \PY{p}{(}\PY{l+m+mi}{86}\PY{p}{,} \PY{l+m+mi}{37}\PY{p}{)}\PY{p}{,} \PY{c+c1}{\PYZsh{} 1}
    \PY{p}{(}\PY{l+m+mi}{17}\PY{p}{,} \PY{l+m+mi}{94}\PY{p}{)}\PY{p}{,} \PY{c+c1}{\PYZsh{} 2}
    \PY{p}{(} \PY{l+m+mi}{3}\PY{p}{,} \PY{l+m+mi}{65}\PY{p}{)}\PY{p}{,} \PY{c+c1}{\PYZsh{} 3}
    \PY{p}{(}\PY{l+m+mi}{48}\PY{p}{,} \PY{l+m+mi}{43}\PY{p}{)}\PY{p}{,} \PY{c+c1}{\PYZsh{} 4}
    \PY{p}{(}\PY{l+m+mi}{78}\PY{p}{,} \PY{l+m+mi}{70}\PY{p}{)}\PY{p}{,} \PY{c+c1}{\PYZsh{} 5}
    \PY{p}{(}\PY{l+m+mi}{17}\PY{p}{,} \PY{l+m+mi}{55}\PY{p}{)}\PY{p}{,} \PY{c+c1}{\PYZsh{} 6}
    \PY{p}{(}\PY{l+m+mi}{62}\PY{p}{,} \PY{l+m+mi}{91}\PY{p}{)}\PY{p}{,} \PY{c+c1}{\PYZsh{} 7}
    \PY{p}{(}\PY{l+m+mi}{78}\PY{p}{,} \PY{l+m+mi}{91}\PY{p}{)}  \PY{c+c1}{\PYZsh{} 8}
\PY{p}{)}
\end{Verbatim}
\end{tcolorbox}

    \begin{itemize}
\tightlist
\item
  \(D=(d_{ij})\), \(i,j \in C\) \(\leftarrow\) matrix of distances
  between cities.
\end{itemize}

    \begin{tcolorbox}[breakable, size=fbox, boxrule=1pt, pad at break*=1mm,colback=cellbackground, colframe=cellborder]
\prompt{In}{incolor}{3}{\boxspacing}
\begin{Verbatim}[commandchars=\\\{\}]
\PY{k+kn}{import} \PY{n+nn}{math}
\PY{n}{distances} \PY{o}{=} \PY{p}{[}
    \PY{p}{[}
        \PY{n+nb}{int}\PY{p}{(}\PY{n}{math}\PY{o}{.}\PY{n}{hypot}\PY{p}{(}\PY{n+nb}{abs}\PY{p}{(}\PY{n}{coords}\PY{p}{[}\PY{n}{i}\PY{p}{]}\PY{p}{[}\PY{l+m+mi}{0}\PY{p}{]} \PY{o}{\PYZhy{}} \PY{n}{coords}\PY{p}{[}\PY{n}{j}\PY{p}{]}\PY{p}{[}\PY{l+m+mi}{0}\PY{p}{]}\PY{p}{)}\PY{p}{,} \PY{n+nb}{abs}\PY{p}{(}\PY{n}{coords}\PY{p}{[}\PY{n}{i}\PY{p}{]}\PY{p}{[}\PY{l+m+mi}{1}\PY{p}{]} \PY{o}{\PYZhy{}} \PY{n}{coords}\PY{p}{[}\PY{n}{j}\PY{p}{]}\PY{p}{[}\PY{l+m+mi}{1}\PY{p}{]}\PY{p}{)}\PY{p}{)}\PY{p}{)}
        \PY{k}{if} \PY{n}{i} \PY{o}{!=} \PY{n}{j} \PY{k}{else} \PY{l+m+mi}{0}
        \PY{k}{for} \PY{n}{j} \PY{o+ow}{in} \PY{n+nb}{range}\PY{p}{(}\PY{n+nb}{len}\PY{p}{(}\PY{n}{coords}\PY{p}{)}\PY{p}{)}
    \PY{p}{]}
    \PY{k}{for} \PY{n}{i} \PY{o+ow}{in} \PY{n+nb}{range}\PY{p}{(}\PY{n+nb}{len}\PY{p}{(}\PY{n}{coords}\PY{p}{)}\PY{p}{)}
\PY{p}{]}
\PY{n}{distances}
\end{Verbatim}
\end{tcolorbox}

            \begin{tcolorbox}[breakable, size=fbox, boxrule=.5pt, pad at break*=1mm, opacityfill=0]
\prompt{Out}{outcolor}{3}{\boxspacing}
\begin{Verbatim}[commandchars=\\\{\}]
[[0, 89, 87, 38, 33, 71, 59, 54],
 [89, 0, 32, 59, 65, 39, 45, 61],
 [87, 32, 0, 50, 75, 17, 64, 79],
 [38, 59, 50, 0, 40, 33, 50, 56],
 [33, 65, 75, 40, 0, 62, 26, 21],
 [71, 39, 17, 33, 62, 0, 57, 70],
 [59, 45, 64, 50, 26, 57, 0, 16],
 [54, 61, 79, 56, 21, 70, 16, 0]]
\end{Verbatim}
\end{tcolorbox}
        
    \hypertarget{objective}{%
\section{Objective}\label{objective}}

The objective is to get a tour \(T\) that visits all cities in \(C\),
ending with the first city, such that the total distance of the tour is
as minimal as possible.

    \[
\begin{aligned}
& \underset{t \in T}{\text{min}}
& & \sum_{i=1}^{n} d_{t_i t_{i+1}}
\end{aligned}
\]

    We already have a feasible solution, our current tour \(T\), so we are
going to use the \emph{2-opt} heuristic to try to improve the tour
according to the objective function, that is, to reduce the total
distance.

Let's create a function to calculate the total cost (distance) of the
tour:

    \begin{tcolorbox}[breakable, size=fbox, boxrule=1pt, pad at break*=1mm,colback=cellbackground, colframe=cellborder]
\prompt{In}{incolor}{4}{\boxspacing}
\begin{Verbatim}[commandchars=\\\{\}]
\PY{k}{def} \PY{n+nf}{dist}\PY{p}{(}\PY{p}{)}\PY{p}{:}
    \PY{k}{return} \PY{n+nb}{sum}\PY{p}{(}\PY{n}{distances}\PY{p}{[}\PY{n}{tour}\PY{p}{[}\PY{n}{i}\PY{p}{]} \PY{o}{\PYZhy{}} \PY{l+m+mi}{1}\PY{p}{]}\PY{p}{[}\PY{n}{tour}\PY{p}{[}\PY{n}{i} \PY{o}{+} \PY{l+m+mi}{1}\PY{p}{]} \PY{o}{\PYZhy{}} \PY{l+m+mi}{1}\PY{p}{]} \PY{k}{for} \PY{n}{i} \PY{o+ow}{in} \PY{n+nb}{range}\PY{p}{(}\PY{n+nb}{len}\PY{p}{(}\PY{n}{tour}\PY{p}{)} \PY{o}{\PYZhy{}} \PY{l+m+mi}{1}\PY{p}{)}\PY{p}{)} \PY{o}{+} \PY{n}{distances}\PY{p}{[}\PY{n}{tour}\PY{p}{[}\PY{o}{\PYZhy{}}\PY{l+m+mi}{1}\PY{p}{]} \PY{o}{\PYZhy{}} \PY{l+m+mi}{1}\PY{p}{]}\PY{p}{[}\PY{n}{tour}\PY{p}{[}\PY{l+m+mi}{0}\PY{p}{]} \PY{o}{\PYZhy{}} \PY{l+m+mi}{1}\PY{p}{]}
\end{Verbatim}
\end{tcolorbox}

    \hypertarget{plotting}{%
\section{Plotting}\label{plotting}}

In order to illustrate each selected neighbor, we will declare a
function to make the corresponding plot for every iteration:

    \begin{tcolorbox}[breakable, size=fbox, boxrule=1pt, pad at break*=1mm,colback=cellbackground, colframe=cellborder]
\prompt{In}{incolor}{5}{\boxspacing}
\begin{Verbatim}[commandchars=\\\{\}]
\PY{k+kn}{import} \PY{n+nn}{matplotlib.pyplot} \PY{k+kn}{as} \PY{n+nn}{plt}
\PY{k}{def} \PY{n+nf}{plot\PYZus{}tour}\PY{p}{(}\PY{p}{)}\PY{p}{:}
    \PY{l+s+sd}{\PYZsq{}\PYZsq{}\PYZsq{}Plots the tour.\PYZsq{}\PYZsq{}\PYZsq{}}
    \PY{n}{x\PYZus{}coords} \PY{o}{=} \PY{p}{[}\PY{n}{c}\PY{p}{[}\PY{l+m+mi}{0}\PY{p}{]} \PY{k}{for} \PY{n}{c} \PY{o+ow}{in} \PY{n}{coords}\PY{p}{]}
    \PY{n}{y\PYZus{}coords} \PY{o}{=} \PY{p}{[}\PY{n}{c}\PY{p}{[}\PY{l+m+mi}{1}\PY{p}{]} \PY{k}{for} \PY{n}{c} \PY{o+ow}{in} \PY{n}{coords}\PY{p}{]}
    \PY{n}{plt}\PY{o}{.}\PY{n}{scatter}\PY{p}{(}\PY{n}{x\PYZus{}coords}\PY{p}{,} \PY{n}{y\PYZus{}coords}\PY{p}{)}
    \PY{k}{for} \PY{n}{city}\PY{p}{,} \PY{n}{tag} \PY{o+ow}{in} \PY{n+nb}{enumerate}\PY{p}{(}\PY{n}{i}\PY{o}{+}\PY{l+m+mi}{1} \PY{k}{for} \PY{n}{i} \PY{o+ow}{in} \PY{n+nb}{range}\PY{p}{(}\PY{l+m+mi}{8}\PY{p}{)}\PY{p}{)}\PY{p}{:}
        \PY{n}{plt}\PY{o}{.}\PY{n}{annotate}\PY{p}{(}\PY{n}{tag}\PY{p}{,} \PY{p}{(}\PY{n}{x\PYZus{}coords}\PY{p}{[}\PY{n}{city}\PY{p}{]}\PY{p}{,} \PY{n}{y\PYZus{}coords}\PY{p}{[}\PY{n}{city}\PY{p}{]}\PY{p}{)}\PY{p}{)}
    \PY{k}{global} \PY{n}{tour}
    \PY{n}{sorted\PYZus{}coords} \PY{o}{=} \PY{p}{[}\PY{n}{coords}\PY{p}{[}\PY{n}{k}\PY{p}{]} \PY{k}{for} \PY{n}{k} \PY{o+ow}{in} \PY{p}{(}\PY{n}{t} \PY{o}{\PYZhy{}} \PY{l+m+mi}{1} \PY{k}{for} \PY{n}{t} \PY{o+ow}{in} \PY{n}{tour} \PY{o}{+} \PY{p}{[}\PY{n}{tour}\PY{p}{[}\PY{l+m+mi}{0}\PY{p}{]}\PY{p}{]}\PY{p}{)}\PY{p}{]}
    \PY{n}{x\PYZus{}values} \PY{o}{=} \PY{p}{[}\PY{n}{c}\PY{p}{[}\PY{l+m+mi}{0}\PY{p}{]} \PY{k}{for} \PY{n}{c} \PY{o+ow}{in} \PY{n}{sorted\PYZus{}coords}\PY{p}{]}
    \PY{n}{y\PYZus{}values} \PY{o}{=} \PY{p}{[}\PY{n}{c}\PY{p}{[}\PY{l+m+mi}{1}\PY{p}{]} \PY{k}{for} \PY{n}{c} \PY{o+ow}{in} \PY{n}{sorted\PYZus{}coords}\PY{p}{]}
    \PY{n}{plt}\PY{o}{.}\PY{n}{plot}\PY{p}{(}\PY{n}{x\PYZus{}values}\PY{p}{,} \PY{n}{y\PYZus{}values}\PY{p}{)}
    \PY{n}{plt}\PY{o}{.}\PY{n}{show}\PY{p}{(}\PY{p}{)}

\PY{o}{\PYZpc{}}\PY{n}{matplotlib} \PY{n}{inline}
\PY{n}{plot\PYZus{}tour}\PY{p}{(}\PY{p}{)}
\end{Verbatim}
\end{tcolorbox}

    \begin{center}
    \adjustimage{max size={0.9\linewidth}{0.9\paperheight}}{homework4_files/homework4_11_0.pdf}
    \end{center}
    { \hspace*{\fill} \\}
    
    \hypertarget{opt}{%
\section{\texorpdfstring{\emph{2-opt}}{2-opt}}\label{opt}}

The \emph{2-opt} algorithm is a local search heuristic that consist in
selecting two edges and swapping their nodes, e.g.
\((a,b),(c,d) \rightarrow (a,c),(b,d)\). If the resulting tour has a
lower cost, then it becomes the current solution. This is repeated until
no improvement has been made.

Let's create a function to reverse the order of the cities between
indexes \(b\) and \(c\) (following the example above), including both.
This is the same as swapping the nodes of two edges:

    \begin{tcolorbox}[breakable, size=fbox, boxrule=1pt, pad at break*=1mm,colback=cellbackground, colframe=cellborder]
\prompt{In}{incolor}{6}{\boxspacing}
\begin{Verbatim}[commandchars=\\\{\}]
\PY{k}{def} \PY{n+nf}{swap}\PY{p}{(}\PY{n}{b}\PY{p}{,} \PY{n}{c}\PY{p}{)}\PY{p}{:}
    \PY{l+s+sd}{\PYZsq{}\PYZsq{}\PYZsq{}Performs the 2\PYZhy{}opt algorithm.\PYZsq{}\PYZsq{}\PYZsq{}}
    \PY{n}{subtour} \PY{o}{=} \PY{n}{tour}\PY{p}{[}\PY{n}{b}\PY{p}{:}\PY{n}{c}\PY{o}{+}\PY{l+m+mi}{1}\PY{p}{]}
    \PY{k}{return} \PY{n}{tour}\PY{p}{[}\PY{p}{:}\PY{n}{b}\PY{p}{]} \PY{o}{+} \PY{n}{subtour}\PY{p}{[}\PY{p}{:}\PY{p}{:}\PY{o}{\PYZhy{}}\PY{l+m+mi}{1}\PY{p}{]} \PY{o}{+} \PY{n}{tour}\PY{p}{[}\PY{n}{c}\PY{o}{+}\PY{l+m+mi}{1}\PY{p}{:}\PY{p}{]}
\end{Verbatim}
\end{tcolorbox}

    The neighborhood of the current tour is formed by every possible
combination of swapping non-adjacent edge pairs. If we don't take into
account the neighbors that have the same order as the tour, the size of
the neighborhood is given by the formula: \[\frac{n(n-3)}{2}\] where
\(n\) is the number of cities to visit. In this instance that number is
\(8\), so each possible tour has \(20\) neighbors.

Now let's create a function to calculate and print the difference in
cost of a neighbor against the current tour, that is, \(\Delta x\).

    \begin{tcolorbox}[breakable, size=fbox, boxrule=1pt, pad at break*=1mm,colback=cellbackground, colframe=cellborder]
\prompt{In}{incolor}{7}{\boxspacing}
\begin{Verbatim}[commandchars=\\\{\}]
\PY{k}{def} \PY{n+nf}{show\PYZus{}dx}\PY{p}{(}\PY{n}{swapped}\PY{p}{,} \PY{n}{b}\PY{p}{,} \PY{n}{c}\PY{p}{)}\PY{p}{:}
    \PY{l+s+sd}{\PYZsq{}\PYZsq{}\PYZsq{}Shows delta X of swapped tour.\PYZsq{}\PYZsq{}\PYZsq{}}
    \PY{n}{a} \PY{o}{=} \PY{l+m+mi}{7} \PY{k}{if} \PY{n}{b} \PY{o}{==} \PY{l+m+mi}{0} \PY{k}{else} \PY{n}{b} \PY{o}{\PYZhy{}} \PY{l+m+mi}{1}
    \PY{n}{d} \PY{o}{=} \PY{l+m+mi}{0} \PY{k}{if} \PY{n}{c} \PY{o}{==} \PY{l+m+mi}{7} \PY{k}{else} \PY{n}{c} \PY{o}{+} \PY{l+m+mi}{1}
    \PY{n}{removed\PYZus{}edges} \PY{o}{=} \PY{n}{distances}\PY{p}{[}\PY{n}{tour}\PY{p}{[}\PY{n}{a}\PY{p}{]} \PY{o}{\PYZhy{}} \PY{l+m+mi}{1}\PY{p}{]}\PY{p}{[}\PY{n}{tour}\PY{p}{[}\PY{n}{b}\PY{p}{]} \PY{o}{\PYZhy{}} \PY{l+m+mi}{1}\PY{p}{]} \PY{o}{+} \PY{n}{distances}\PY{p}{[}\PY{n}{tour}\PY{p}{[}\PY{n}{c}\PY{p}{]} \PY{o}{\PYZhy{}} \PY{l+m+mi}{1}\PY{p}{]}\PY{p}{[}\PY{n}{tour}\PY{p}{[}\PY{n}{d}\PY{p}{]} \PY{o}{\PYZhy{}} \PY{l+m+mi}{1}\PY{p}{]}
    \PY{n}{new\PYZus{}edges} \PY{o}{=} \PY{n}{distances}\PY{p}{[}\PY{n}{swapped}\PY{p}{[}\PY{n}{a}\PY{p}{]} \PY{o}{\PYZhy{}} \PY{l+m+mi}{1}\PY{p}{]}\PY{p}{[}\PY{n}{swapped}\PY{p}{[}\PY{n}{b}\PY{p}{]} \PY{o}{\PYZhy{}} \PY{l+m+mi}{1}\PY{p}{]} \PY{o}{+} \PY{n}{distances}\PY{p}{[}\PY{n}{swapped}\PY{p}{[}\PY{n}{c}\PY{p}{]} \PY{o}{\PYZhy{}} \PY{l+m+mi}{1}\PY{p}{]}\PY{p}{[}\PY{n}{swapped}\PY{p}{[}\PY{n}{d}\PY{p}{]} \PY{o}{\PYZhy{}} \PY{l+m+mi}{1}\PY{p}{]}
    \PY{n}{dx} \PY{o}{=} \PY{n}{new\PYZus{}edges} \PY{o}{\PYZhy{}} \PY{n}{removed\PYZus{}edges}
    \PY{k}{print}\PY{p}{(}\PY{n}{f}\PY{l+s+s1}{\PYZsq{}}\PY{l+s+s1}{\PYZob{}swapped\PYZcb{} \PYZob{}dx:7g\PYZcb{}}\PY{l+s+s1}{\PYZsq{}}\PY{p}{)}
    \PY{k}{return} \PY{n}{dx}
\end{Verbatim}
\end{tcolorbox}

    The lower the \(\Delta x\), the better the neighbor, because it
indicates how much units of distance that neighbor adds, so if it's a
negative value then it's decreasing the total cost.

    \hypertarget{best-found-strategy}{%
\subsection{Best Found Strategy}\label{best-found-strategy}}

The BFS consists in calculating the \(\Delta x\) for every neighbor and
selecting the one that makes the best improvement.

Let's define a function to show all the neighbors and calculate their
\(\Delta x\). This function should print the 20 neighbors of each tour:

    \begin{tcolorbox}[breakable, size=fbox, boxrule=1pt, pad at break*=1mm,colback=cellbackground, colframe=cellborder]
\prompt{In}{incolor}{8}{\boxspacing}
\begin{Verbatim}[commandchars=\\\{\}]
\PY{k}{def} \PY{n+nf}{bfs}\PY{p}{(}\PY{p}{)}\PY{p}{:}
    \PY{l+s+sd}{\PYZsq{}\PYZsq{}\PYZsq{}Shows all the neighbors of the current tour.\PYZsq{}\PYZsq{}\PYZsq{}}
    \PY{k}{for} \PY{n}{i} \PY{o+ow}{in} \PY{n+nb}{range}\PY{p}{(}\PY{n+nb}{len}\PY{p}{(}\PY{n}{tour}\PY{p}{)}\PY{p}{)}\PY{p}{:}
        \PY{k}{for} \PY{n}{j} \PY{o+ow}{in} \PY{n+nb}{range}\PY{p}{(}\PY{n}{i} \PY{o}{+} \PY{l+m+mi}{1}\PY{p}{,} \PY{n+nb}{len}\PY{p}{(}\PY{n}{tour}\PY{p}{)} \PY{o}{\PYZhy{}} \PY{l+m+mi}{1}\PY{p}{)}\PY{p}{:}
            \PY{n}{neighbor} \PY{o}{=} \PY{n}{swap}\PY{p}{(}\PY{n}{i}\PY{p}{,} \PY{n}{j}\PY{p}{)}
            \PY{n}{show\PYZus{}dx}\PY{p}{(}\PY{n}{neighbor}\PY{p}{,} \PY{n}{i}\PY{p}{,} \PY{n}{j}\PY{p}{)}
\end{Verbatim}
\end{tcolorbox}

    \hypertarget{first-iteration}{%
\subsubsection{First iteration}\label{first-iteration}}

We will call the BFS function to see which neighbor of \(T\) makes the
best improvement:

    \begin{tcolorbox}[breakable, size=fbox, boxrule=1pt, pad at break*=1mm,colback=cellbackground, colframe=cellborder]
\prompt{In}{incolor}{9}{\boxspacing}
\begin{Verbatim}[commandchars=\\\{\}]
\PY{n}{bfs}\PY{p}{(}\PY{p}{)}
\end{Verbatim}
\end{tcolorbox}

    \begin{Verbatim}[commandchars=\\\{\}]
[2, 1, 3, 4, 5, 6, 7, 8]      62
[3, 2, 1, 4, 5, 6, 7, 8]      13
[4, 3, 2, 1, 5, 6, 7, 8]      -5
[5, 4, 3, 2, 1, 6, 7, 8]     -24
[6, 5, 4, 3, 2, 1, 7, 8]      18
[7, 6, 5, 4, 3, 2, 1, 8]       0
[1, 3, 2, 4, 5, 6, 7, 8]       7
[1, 4, 3, 2, 5, 6, 7, 8]     -26
[1, 5, 4, 3, 2, 6, 7, 8]     -79
[1, 6, 5, 4, 3, 2, 7, 8]     -30
[1, 7, 6, 5, 4, 3, 2, 8]      15
[1, 2, 4, 3, 5, 6, 7, 8]      62
[1, 2, 5, 4, 3, 6, 7, 8]     -12
[1, 2, 6, 5, 4, 3, 7, 8]      14
[1, 2, 7, 6, 5, 4, 3, 8]      76
[1, 2, 3, 5, 4, 6, 7, 8]      -4
[1, 2, 3, 6, 5, 4, 7, 8]     -40
[1, 2, 3, 7, 6, 5, 4, 8]      54
[1, 2, 3, 4, 6, 5, 7, 8]     -38
[1, 2, 3, 4, 7, 6, 5, 8]      15
[1, 2, 3, 4, 5, 7, 6, 8]      18
    \end{Verbatim}

    As we can see, the lowest \(\Delta x\) reduces total cost by \(79\)
units, so we move the solution to its corresponding tour:
\(T \leftarrow (1, 5, 4, 3, 2, 6, 7, 8)\).

    \begin{tcolorbox}[breakable, size=fbox, boxrule=1pt, pad at break*=1mm,colback=cellbackground, colframe=cellborder]
\prompt{In}{incolor}{10}{\boxspacing}
\begin{Verbatim}[commandchars=\\\{\}]
\PY{n}{tour} \PY{o}{=} \PY{p}{[}\PY{l+m+mi}{1}\PY{p}{,} \PY{l+m+mi}{5}\PY{p}{,} \PY{l+m+mi}{4}\PY{p}{,} \PY{l+m+mi}{3}\PY{p}{,} \PY{l+m+mi}{2}\PY{p}{,} \PY{l+m+mi}{6}\PY{p}{,} \PY{l+m+mi}{7}\PY{p}{,} \PY{l+m+mi}{8}\PY{p}{]}
\PY{n}{plot\PYZus{}tour}\PY{p}{(}\PY{p}{)}
\end{Verbatim}
\end{tcolorbox}

    \begin{center}
    \adjustimage{max size={0.9\linewidth}{0.9\paperheight}}{homework4_files/homework4_22_0.pdf}
    \end{center}
    { \hspace*{\fill} \\}
    
    We can observe that edges \((1,2)\) and \((5,6)\) were swapped.

    \hypertarget{second-iteration}{%
\subsubsection{Second iteration}\label{second-iteration}}

Let's see if we can improve the tour:

    \begin{tcolorbox}[breakable, size=fbox, boxrule=1pt, pad at break*=1mm,colback=cellbackground, colframe=cellborder]
\prompt{In}{incolor}{11}{\boxspacing}
\begin{Verbatim}[commandchars=\\\{\}]
\PY{n}{bfs}\PY{p}{(}\PY{p}{)}
\end{Verbatim}
\end{tcolorbox}

    \begin{Verbatim}[commandchars=\\\{\}]
[5, 1, 4, 3, 2, 6, 7, 8]     -35
[4, 5, 1, 3, 2, 6, 7, 8]      39
[3, 4, 5, 1, 2, 6, 7, 8]      82
[2, 3, 4, 5, 1, 6, 7, 8]      39
[6, 2, 3, 4, 5, 1, 7, 8]      18
[7, 6, 2, 3, 4, 5, 1, 8]       0
[1, 4, 5, 3, 2, 6, 7, 8]      30
[1, 3, 4, 5, 2, 6, 7, 8]      87
[1, 2, 3, 4, 5, 6, 7, 8]      79
[1, 6, 2, 3, 4, 5, 7, 8]       7
[1, 7, 6, 2, 3, 4, 5, 8]      31
[1, 5, 3, 4, 2, 6, 7, 8]      62
[1, 5, 2, 3, 4, 6, 7, 8]      19
[1, 5, 6, 2, 3, 4, 7, 8]      15
[1, 5, 7, 6, 2, 3, 4, 8]      26
[1, 5, 4, 2, 3, 6, 7, 8]     -13
[1, 5, 4, 6, 2, 3, 7, 8]     -10
[1, 5, 4, 7, 6, 2, 3, 8]      63
[1, 5, 4, 3, 6, 2, 7, 8]     -27
[1, 5, 4, 3, 7, 6, 2, 8]      77
[1, 5, 4, 3, 2, 7, 6, 8]      60
    \end{Verbatim}

    Indeed we can, the lowest \(\Delta x\) is now \(-35\). We can observe
that only a few neighbors are better than the tour this time. The
solution moves to \((5, 1, 4, 3, 2, 6, 7, 8)\).

    \begin{tcolorbox}[breakable, size=fbox, boxrule=1pt, pad at break*=1mm,colback=cellbackground, colframe=cellborder]
\prompt{In}{incolor}{12}{\boxspacing}
\begin{Verbatim}[commandchars=\\\{\}]
\PY{n}{tour} \PY{o}{=} \PY{p}{[}\PY{l+m+mi}{5}\PY{p}{,} \PY{l+m+mi}{1}\PY{p}{,} \PY{l+m+mi}{4}\PY{p}{,} \PY{l+m+mi}{3}\PY{p}{,} \PY{l+m+mi}{2}\PY{p}{,} \PY{l+m+mi}{6}\PY{p}{,} \PY{l+m+mi}{7}\PY{p}{,} \PY{l+m+mi}{8}\PY{p}{]}
\PY{n}{plot\PYZus{}tour}\PY{p}{(}\PY{p}{)}
\end{Verbatim}
\end{tcolorbox}

    \begin{center}
    \adjustimage{max size={0.9\linewidth}{0.9\paperheight}}{homework4_files/homework4_27_0.pdf}
    \end{center}
    { \hspace*{\fill} \\}
    
    In this iteration, edges \((8,1)\) and \((5,4)\) were swapped.

    \hypertarget{third-iteration}{%
\subsubsection{Third iteration}\label{third-iteration}}

Let's check again the best neighbor by calling the BFS function:

    \begin{tcolorbox}[breakable, size=fbox, boxrule=1pt, pad at break*=1mm,colback=cellbackground, colframe=cellborder]
\prompt{In}{incolor}{13}{\boxspacing}
\begin{Verbatim}[commandchars=\\\{\}]
\PY{n}{bfs}\PY{p}{(}\PY{p}{)}
\end{Verbatim}
\end{tcolorbox}

    \begin{Verbatim}[commandchars=\\\{\}]
[1, 5, 4, 3, 2, 6, 7, 8]      35
[4, 1, 5, 3, 2, 6, 7, 8]      60
[3, 4, 1, 5, 2, 6, 7, 8]      91
[2, 3, 4, 1, 5, 6, 7, 8]      63
[6, 2, 3, 4, 1, 5, 7, 8]      18
[7, 6, 2, 3, 4, 1, 5, 8]       0
[5, 4, 1, 3, 2, 6, 7, 8]      44
[5, 3, 4, 1, 2, 6, 7, 8]      99
[5, 2, 3, 4, 1, 6, 7, 8]      64
[5, 6, 2, 3, 4, 1, 7, 8]      31
[5, 7, 6, 2, 3, 4, 1, 8]      31
[5, 1, 3, 4, 2, 6, 7, 8]      76
[5, 1, 2, 3, 4, 6, 7, 8]      45
[5, 1, 6, 2, 3, 4, 7, 8]      26
[5, 1, 7, 6, 2, 3, 4, 8]      61
[5, 1, 4, 2, 3, 6, 7, 8]     -13
[5, 1, 4, 6, 2, 3, 7, 8]     -10
[5, 1, 4, 7, 6, 2, 3, 8]      63
[5, 1, 4, 3, 6, 2, 7, 8]     -27
[5, 1, 4, 3, 7, 6, 2, 8]      77
[5, 1, 4, 3, 2, 7, 6, 8]      60
    \end{Verbatim}

    The lowest \(\Delta x\) is \(-27\) so the tour moves to its neighbor:
\(T \leftarrow (5, 1, 4, 3, 6, 2, 7, 8)\).

    \begin{tcolorbox}[breakable, size=fbox, boxrule=1pt, pad at break*=1mm,colback=cellbackground, colframe=cellborder]
\prompt{In}{incolor}{14}{\boxspacing}
\begin{Verbatim}[commandchars=\\\{\}]
\PY{n}{tour} \PY{o}{=} \PY{p}{[}\PY{l+m+mi}{5}\PY{p}{,} \PY{l+m+mi}{1}\PY{p}{,} \PY{l+m+mi}{4}\PY{p}{,} \PY{l+m+mi}{3}\PY{p}{,} \PY{l+m+mi}{6}\PY{p}{,} \PY{l+m+mi}{2}\PY{p}{,} \PY{l+m+mi}{7}\PY{p}{,} \PY{l+m+mi}{8}\PY{p}{]}
\PY{n}{plot\PYZus{}tour}\PY{p}{(}\PY{p}{)}
\end{Verbatim}
\end{tcolorbox}

    \begin{center}
    \adjustimage{max size={0.9\linewidth}{0.9\paperheight}}{homework4_files/homework4_32_0.pdf}
    \end{center}
    { \hspace*{\fill} \\}
    
    We can see that edges \((3,2)\) and \((6,7)\) were swapped.

    \hypertarget{fourth-iteration}{%
\subsubsection{Fourth iteration}\label{fourth-iteration}}

We will choose the best neighbor according to its \(\Delta x\):

    \begin{tcolorbox}[breakable, size=fbox, boxrule=1pt, pad at break*=1mm,colback=cellbackground, colframe=cellborder]
\prompt{In}{incolor}{15}{\boxspacing}
\begin{Verbatim}[commandchars=\\\{\}]
\PY{n}{bfs}\PY{p}{(}\PY{p}{)}
\end{Verbatim}
\end{tcolorbox}

    \begin{Verbatim}[commandchars=\\\{\}]
[1, 5, 4, 3, 6, 2, 7, 8]      35
[4, 1, 5, 3, 6, 2, 7, 8]      60
[3, 4, 1, 5, 6, 2, 7, 8]     103
[6, 3, 4, 1, 5, 2, 7, 8]      75
[2, 6, 3, 4, 1, 5, 7, 8]      21
[7, 2, 6, 3, 4, 1, 5, 8]       0
[5, 4, 1, 3, 6, 2, 7, 8]      44
[5, 3, 4, 1, 6, 2, 7, 8]      96
[5, 6, 3, 4, 1, 2, 7, 8]      79
[5, 2, 6, 3, 4, 1, 7, 8]      46
[5, 7, 2, 6, 3, 4, 1, 8]      31
[5, 1, 3, 4, 6, 2, 7, 8]      65
[5, 1, 6, 3, 4, 2, 7, 8]      53
[5, 1, 2, 6, 3, 4, 7, 8]      56
[5, 1, 7, 2, 6, 3, 4, 8]      61
[5, 1, 4, 6, 3, 2, 7, 8]     -24
[5, 1, 4, 2, 6, 3, 7, 8]      28
[5, 1, 4, 7, 2, 6, 3, 8]      63
[5, 1, 4, 3, 2, 6, 7, 8]      27
[5, 1, 4, 3, 7, 2, 6, 8]     101
[5, 1, 4, 3, 6, 7, 2, 8]      63
    \end{Verbatim}

    We can see that the lowest value of \(\Delta x\) is getting smaller
every iteration. This time it reduces the tour's cost by \(24\) units.
Updating the tour: \(T \leftarrow (5, 1, 4, 6, 3, 2, 7, 8)\).

    \begin{tcolorbox}[breakable, size=fbox, boxrule=1pt, pad at break*=1mm,colback=cellbackground, colframe=cellborder]
\prompt{In}{incolor}{16}{\boxspacing}
\begin{Verbatim}[commandchars=\\\{\}]
\PY{n}{tour} \PY{o}{=} \PY{p}{[}\PY{l+m+mi}{5}\PY{p}{,} \PY{l+m+mi}{1}\PY{p}{,} \PY{l+m+mi}{4}\PY{p}{,} \PY{l+m+mi}{6}\PY{p}{,} \PY{l+m+mi}{3}\PY{p}{,} \PY{l+m+mi}{2}\PY{p}{,} \PY{l+m+mi}{7}\PY{p}{,} \PY{l+m+mi}{8}\PY{p}{]}
\PY{n}{plot\PYZus{}tour}\PY{p}{(}\PY{p}{)}
\end{Verbatim}
\end{tcolorbox}

    \begin{center}
    \adjustimage{max size={0.9\linewidth}{0.9\paperheight}}{homework4_files/homework4_37_0.pdf}
    \end{center}
    { \hspace*{\fill} \\}
    
    Edges \((4,3)\) and \((6,2)\) were swapped this time.

    \hypertarget{fifth-iteration}{%
\subsubsection{Fifth iteration}\label{fifth-iteration}}

Let's try to improve our current solution by moving it to one of its
neighbors:

    \begin{tcolorbox}[breakable, size=fbox, boxrule=1pt, pad at break*=1mm,colback=cellbackground, colframe=cellborder]
\prompt{In}{incolor}{17}{\boxspacing}
\begin{Verbatim}[commandchars=\\\{\}]
\PY{n}{bfs}\PY{p}{(}\PY{p}{)}
\end{Verbatim}
\end{tcolorbox}

    \begin{Verbatim}[commandchars=\\\{\}]
[1, 5, 4, 6, 3, 2, 7, 8]      35
[4, 1, 5, 6, 3, 2, 7, 8]      64
[6, 4, 1, 5, 3, 2, 7, 8]     107
[3, 6, 4, 1, 5, 2, 7, 8]      91
[2, 3, 6, 4, 1, 5, 7, 8]      21
[7, 2, 3, 6, 4, 1, 5, 8]       0
[5, 4, 1, 6, 3, 2, 7, 8]      45
[5, 6, 4, 1, 3, 2, 7, 8]      99
[5, 3, 6, 4, 1, 2, 7, 8]      99
[5, 2, 3, 6, 4, 1, 7, 8]      46
[5, 7, 2, 3, 6, 4, 1, 8]      31
[5, 1, 6, 4, 3, 2, 7, 8]      66
[5, 1, 3, 6, 4, 2, 7, 8]      76
[5, 1, 2, 3, 6, 4, 7, 8]      56
[5, 1, 7, 2, 3, 6, 4, 8]      61
[5, 1, 4, 3, 6, 2, 7, 8]      24
[5, 1, 4, 2, 3, 6, 7, 8]      38
[5, 1, 4, 7, 2, 3, 6, 8]      71
[5, 1, 4, 6, 2, 3, 7, 8]      41
[5, 1, 4, 6, 7, 2, 3, 8]     103
[5, 1, 4, 6, 3, 7, 2, 8]      77
    \end{Verbatim}

    No neighbor can improve the tour, as we can confirm by looking at the
\(\Delta x\) column, none of them is less than 0. This means that we
have reached a local optimal.

Using the Best Found Strategy of the \emph{2-opt} heuristic, we get the
solution \(T = (5, 1, 4, 6, 3, 2, 7, 8)\) after 4 iterations. Let's
calculate its total cost:

    \begin{tcolorbox}[breakable, size=fbox, boxrule=1pt, pad at break*=1mm,colback=cellbackground, colframe=cellborder]
\prompt{In}{incolor}{18}{\boxspacing}
\begin{Verbatim}[commandchars=\\\{\}]
\PY{n}{dist}\PY{p}{(}\PY{p}{)}
\end{Verbatim}
\end{tcolorbox}

            \begin{tcolorbox}[breakable, size=fbox, boxrule=.5pt, pad at break*=1mm, opacityfill=0]
\prompt{Out}{outcolor}{18}{\boxspacing}
\begin{Verbatim}[commandchars=\\\{\}]
235
\end{Verbatim}
\end{tcolorbox}
        
    The local optimal solution \(T\) has a cost of \(235\) units of
distance.

    \begin{tcolorbox}[breakable, size=fbox, boxrule=1pt, pad at break*=1mm,colback=cellbackground, colframe=cellborder]
\prompt{In}{incolor}{19}{\boxspacing}
\begin{Verbatim}[commandchars=\\\{\}]
\PY{n}{plot\PYZus{}tour}\PY{p}{(}\PY{p}{)}
\end{Verbatim}
\end{tcolorbox}

    \begin{center}
    \adjustimage{max size={0.9\linewidth}{0.9\paperheight}}{homework4_files/homework4_44_0.pdf}
    \end{center}
    { \hspace*{\fill} \\}
    
    \hypertarget{first-found-strategy}{%
\subsection{First Found Strategy}\label{first-found-strategy}}

In the FFS, the solution moves to the first neighbor that would improve
the tour. It doesn't matter if there were neighbors left to be
evaluated. As the BFS, this repeats until the neighborhood has been
explored without any improvement.

We are going to use a function that will evaluate a neighbor. If it
improves the solution, no more neighbors will be evaluated, otherwise
the next neighbor will be evaluated, and so on until the whole
neighborhood is explored.

    \begin{tcolorbox}[breakable, size=fbox, boxrule=1pt, pad at break*=1mm,colback=cellbackground, colframe=cellborder]
\prompt{In}{incolor}{20}{\boxspacing}
\begin{Verbatim}[commandchars=\\\{\}]
\PY{k}{def} \PY{n+nf}{ffs}\PY{p}{(}\PY{p}{)}\PY{p}{:}
    \PY{l+s+sd}{\PYZsq{}\PYZsq{}\PYZsq{}Show evaluated neighbors until improvement.\PYZsq{}\PYZsq{}\PYZsq{}}
    \PY{k}{for} \PY{n}{i} \PY{o+ow}{in} \PY{n+nb}{range}\PY{p}{(}\PY{n+nb}{len}\PY{p}{(}\PY{n}{tour}\PY{p}{)}\PY{p}{)}\PY{p}{:}
        \PY{k}{for} \PY{n}{j} \PY{o+ow}{in} \PY{n+nb}{range}\PY{p}{(}\PY{n}{i} \PY{o}{+} \PY{l+m+mi}{1}\PY{p}{,} \PY{n+nb}{len}\PY{p}{(}\PY{n}{tour}\PY{p}{)} \PY{o}{\PYZhy{}} \PY{l+m+mi}{1}\PY{p}{)}\PY{p}{:}
            \PY{n}{neighbor} \PY{o}{=} \PY{n}{swap}\PY{p}{(}\PY{n}{i}\PY{p}{,} \PY{n}{j}\PY{p}{)}
            \PY{k}{if} \PY{n}{show\PYZus{}dx}\PY{p}{(}\PY{n}{neighbor}\PY{p}{,} \PY{n}{i}\PY{p}{,} \PY{n}{j}\PY{p}{)} \PY{o}{\PYZlt{}} \PY{l+m+mi}{0}\PY{p}{:}
                \PY{k}{return}
\end{Verbatim}
\end{tcolorbox}

    We can observe that the function is almost identical to \texttt{bfs()}.
The only difference is that \texttt{ffs()} finishes earlier if an
improvement is found.

Before using the FFS, let's restart our tour as we had it before using
the BFS: \(T \leftarrow (1,2,3,4,5,6,7,8)\).

    \begin{tcolorbox}[breakable, size=fbox, boxrule=1pt, pad at break*=1mm,colback=cellbackground, colframe=cellborder]
\prompt{In}{incolor}{21}{\boxspacing}
\begin{Verbatim}[commandchars=\\\{\}]
\PY{n}{tour} \PY{o}{=} \PY{p}{[}\PY{l+m+mi}{1}\PY{p}{,} \PY{l+m+mi}{2}\PY{p}{,} \PY{l+m+mi}{3}\PY{p}{,} \PY{l+m+mi}{4}\PY{p}{,} \PY{l+m+mi}{5}\PY{p}{,} \PY{l+m+mi}{6}\PY{p}{,} \PY{l+m+mi}{7}\PY{p}{,} \PY{l+m+mi}{8}\PY{p}{]}
\PY{n}{plot\PYZus{}tour}\PY{p}{(}\PY{p}{)}
\end{Verbatim}
\end{tcolorbox}

    \begin{center}
    \adjustimage{max size={0.9\linewidth}{0.9\paperheight}}{homework4_files/homework4_48_0.pdf}
    \end{center}
    { \hspace*{\fill} \\}
    
    \hypertarget{first-iteration}{%
\subsubsection{First iteration}\label{first-iteration}}

Let's see how many neighbors are evaluated before an improvement is
found:

    \begin{tcolorbox}[breakable, size=fbox, boxrule=1pt, pad at break*=1mm,colback=cellbackground, colframe=cellborder]
\prompt{In}{incolor}{22}{\boxspacing}
\begin{Verbatim}[commandchars=\\\{\}]
\PY{n}{ffs}\PY{p}{(}\PY{p}{)}
\end{Verbatim}
\end{tcolorbox}

    \begin{Verbatim}[commandchars=\\\{\}]
[2, 1, 3, 4, 5, 6, 7, 8]      62
[3, 2, 1, 4, 5, 6, 7, 8]      13
[4, 3, 2, 1, 5, 6, 7, 8]      -5
    \end{Verbatim}

    Only two and then the improvement. It's a small improvement,
\(\Delta x\) only reduces the cost by \(5\) units, but very few
evaluations were made. Now we move our solution:
\(T \leftarrow (4, 3, 2, 1, 5, 6, 7, 8)\). At this point, the local
search using FFS takes a different movement from BFS.

    \begin{tcolorbox}[breakable, size=fbox, boxrule=1pt, pad at break*=1mm,colback=cellbackground, colframe=cellborder]
\prompt{In}{incolor}{23}{\boxspacing}
\begin{Verbatim}[commandchars=\\\{\}]
\PY{n}{tour} \PY{o}{=} \PY{p}{[}\PY{l+m+mi}{4}\PY{p}{,} \PY{l+m+mi}{3}\PY{p}{,} \PY{l+m+mi}{2}\PY{p}{,} \PY{l+m+mi}{1}\PY{p}{,} \PY{l+m+mi}{5}\PY{p}{,} \PY{l+m+mi}{6}\PY{p}{,} \PY{l+m+mi}{7}\PY{p}{,} \PY{l+m+mi}{8}\PY{p}{]}
\PY{n}{plot\PYZus{}tour}\PY{p}{(}\PY{p}{)}
\end{Verbatim}
\end{tcolorbox}

    \begin{center}
    \adjustimage{max size={0.9\linewidth}{0.9\paperheight}}{homework4_files/homework4_52_0.pdf}
    \end{center}
    { \hspace*{\fill} \\}
    
    Edges \((8,1)\) and \((4,5)\) were swapped.

    \hypertarget{second-iteration}{%
\subsubsection{Second iteration}\label{second-iteration}}

Let's evaluate the neighbors of \(T\) so we can improve it:

    \begin{tcolorbox}[breakable, size=fbox, boxrule=1pt, pad at break*=1mm,colback=cellbackground, colframe=cellborder]
\prompt{In}{incolor}{24}{\boxspacing}
\begin{Verbatim}[commandchars=\\\{\}]
\PY{n}{ffs}\PY{p}{(}\PY{p}{)}
\end{Verbatim}
\end{tcolorbox}

    \begin{Verbatim}[commandchars=\\\{\}]
[3, 4, 2, 1, 5, 6, 7, 8]      50
[2, 3, 4, 1, 5, 6, 7, 8]     -46
    \end{Verbatim}

    The better neighbor has a \(\Delta x\) of \(-46\) and was quickly found.
This is a far better improvement than the previous one. The tour moves
to that neighbor: \(T \leftarrow (2, 3, 4, 1, 5, 6, 7, 8)\).

    \begin{tcolorbox}[breakable, size=fbox, boxrule=1pt, pad at break*=1mm,colback=cellbackground, colframe=cellborder]
\prompt{In}{incolor}{25}{\boxspacing}
\begin{Verbatim}[commandchars=\\\{\}]
\PY{n}{tour} \PY{o}{=} \PY{p}{[}\PY{l+m+mi}{2}\PY{p}{,} \PY{l+m+mi}{3}\PY{p}{,} \PY{l+m+mi}{4}\PY{p}{,} \PY{l+m+mi}{1}\PY{p}{,} \PY{l+m+mi}{5}\PY{p}{,} \PY{l+m+mi}{6}\PY{p}{,} \PY{l+m+mi}{7}\PY{p}{,} \PY{l+m+mi}{8}\PY{p}{]}
\PY{n}{plot\PYZus{}tour}\PY{p}{(}\PY{p}{)}
\end{Verbatim}
\end{tcolorbox}

    \begin{center}
    \adjustimage{max size={0.9\linewidth}{0.9\paperheight}}{homework4_files/homework4_57_0.pdf}
    \end{center}
    { \hspace*{\fill} \\}
    
    We can see that the algorithm swapped edges \((8,4)\) and \((2,1)\).

    \hypertarget{third-iteration}{%
\subsubsection{Third iteration}\label{third-iteration}}

We will again search for a better solution by calling the FFS function:

    \begin{tcolorbox}[breakable, size=fbox, boxrule=1pt, pad at break*=1mm,colback=cellbackground, colframe=cellborder]
\prompt{In}{incolor}{26}{\boxspacing}
\begin{Verbatim}[commandchars=\\\{\}]
\PY{n}{ffs}\PY{p}{(}\PY{p}{)}
\end{Verbatim}
\end{tcolorbox}

    \begin{Verbatim}[commandchars=\\\{\}]
[3, 2, 4, 1, 5, 6, 7, 8]      27
[4, 3, 2, 1, 5, 6, 7, 8]      46
[1, 4, 3, 2, 5, 6, 7, 8]      25
[5, 1, 4, 3, 2, 6, 7, 8]     -63
    \end{Verbatim}

    The first improvement found reduces the tour cost by \(63\) units. Let's
move the solution to it: \(T \leftarrow (5, 1, 4, 3, 2, 6, 7, 8)\).

    \begin{tcolorbox}[breakable, size=fbox, boxrule=1pt, pad at break*=1mm,colback=cellbackground, colframe=cellborder]
\prompt{In}{incolor}{27}{\boxspacing}
\begin{Verbatim}[commandchars=\\\{\}]
\PY{n}{tour} \PY{o}{=} \PY{p}{[}\PY{l+m+mi}{5}\PY{p}{,} \PY{l+m+mi}{1}\PY{p}{,} \PY{l+m+mi}{4}\PY{p}{,} \PY{l+m+mi}{3}\PY{p}{,} \PY{l+m+mi}{2}\PY{p}{,} \PY{l+m+mi}{6}\PY{p}{,} \PY{l+m+mi}{7}\PY{p}{,} \PY{l+m+mi}{8}\PY{p}{]}
\PY{n}{plot\PYZus{}tour}\PY{p}{(}\PY{p}{)}
\end{Verbatim}
\end{tcolorbox}

    \begin{center}
    \adjustimage{max size={0.9\linewidth}{0.9\paperheight}}{homework4_files/homework4_62_0.pdf}
    \end{center}
    { \hspace*{\fill} \\}
    
    This time, edges \((8,2)\) and \((5,6)\) were swapped.

    \hypertarget{fourth-iteration}{%
\subsubsection{Fourth iteration}\label{fourth-iteration}}

Let's see if we can still improve the solution:

    \begin{tcolorbox}[breakable, size=fbox, boxrule=1pt, pad at break*=1mm,colback=cellbackground, colframe=cellborder]
\prompt{In}{incolor}{28}{\boxspacing}
\begin{Verbatim}[commandchars=\\\{\}]
\PY{n}{ffs}\PY{p}{(}\PY{p}{)}
\end{Verbatim}
\end{tcolorbox}

    \begin{Verbatim}[commandchars=\\\{\}]
[1, 5, 4, 3, 2, 6, 7, 8]      35
[4, 1, 5, 3, 2, 6, 7, 8]      60
[3, 4, 1, 5, 2, 6, 7, 8]      91
[2, 3, 4, 1, 5, 6, 7, 8]      63
[6, 2, 3, 4, 1, 5, 7, 8]      18
[7, 6, 2, 3, 4, 1, 5, 8]       0
[5, 4, 1, 3, 2, 6, 7, 8]      44
[5, 3, 4, 1, 2, 6, 7, 8]      99
[5, 2, 3, 4, 1, 6, 7, 8]      64
[5, 6, 2, 3, 4, 1, 7, 8]      31
[5, 7, 6, 2, 3, 4, 1, 8]      31
[5, 1, 3, 4, 2, 6, 7, 8]      76
[5, 1, 2, 3, 4, 6, 7, 8]      45
[5, 1, 6, 2, 3, 4, 7, 8]      26
[5, 1, 7, 6, 2, 3, 4, 8]      61
[5, 1, 4, 2, 3, 6, 7, 8]     -13
    \end{Verbatim}

    Of course we can. The chosen neighbor has a \(\Delta x\) of \(-13\). We
can see that this time, the algorithm had to explore more than the past
iterations. The tour now moves to \((5, 1, 4, 2, 3, 6, 7, 8)\).

    \begin{tcolorbox}[breakable, size=fbox, boxrule=1pt, pad at break*=1mm,colback=cellbackground, colframe=cellborder]
\prompt{In}{incolor}{29}{\boxspacing}
\begin{Verbatim}[commandchars=\\\{\}]
\PY{n}{tour} \PY{o}{=} \PY{p}{[}\PY{l+m+mi}{5}\PY{p}{,} \PY{l+m+mi}{1}\PY{p}{,} \PY{l+m+mi}{4}\PY{p}{,} \PY{l+m+mi}{2}\PY{p}{,} \PY{l+m+mi}{3}\PY{p}{,} \PY{l+m+mi}{6}\PY{p}{,} \PY{l+m+mi}{7}\PY{p}{,} \PY{l+m+mi}{8}\PY{p}{]}
\PY{n}{plot\PYZus{}tour}\PY{p}{(}\PY{p}{)}
\end{Verbatim}
\end{tcolorbox}

    \begin{center}
    \adjustimage{max size={0.9\linewidth}{0.9\paperheight}}{homework4_files/homework4_67_0.pdf}
    \end{center}
    { \hspace*{\fill} \\}
    
    Edges \((4,3)\) and \((2,6)\) were swapped.

    \hypertarget{fifth-iteration}{%
\subsubsection{Fifth iteration}\label{fifth-iteration}}

We will call the FFS function to again explore the neighborhood in
search for an improvement.

    \begin{tcolorbox}[breakable, size=fbox, boxrule=1pt, pad at break*=1mm,colback=cellbackground, colframe=cellborder]
\prompt{In}{incolor}{30}{\boxspacing}
\begin{Verbatim}[commandchars=\\\{\}]
\PY{n}{ffs}\PY{p}{(}\PY{p}{)}
\end{Verbatim}
\end{tcolorbox}

    \begin{Verbatim}[commandchars=\\\{\}]
[1, 5, 4, 2, 3, 6, 7, 8]      35
[4, 1, 5, 2, 3, 6, 7, 8]      41
[2, 4, 1, 5, 3, 6, 7, 8]      83
[3, 2, 4, 1, 5, 6, 7, 8]     103
[6, 3, 2, 4, 1, 5, 7, 8]      18
[7, 6, 3, 2, 4, 1, 5, 8]       0
[5, 4, 1, 2, 3, 6, 7, 8]      37
[5, 2, 4, 1, 3, 6, 7, 8]      87
[5, 3, 2, 4, 1, 6, 7, 8]      96
[5, 6, 3, 2, 4, 1, 7, 8]      31
[5, 7, 6, 3, 2, 4, 1, 8]      31
[5, 1, 2, 4, 3, 6, 7, 8]      69
[5, 1, 3, 2, 4, 6, 7, 8]      65
[5, 1, 6, 3, 2, 4, 7, 8]      26
[5, 1, 7, 6, 3, 2, 4, 8]      61
[5, 1, 4, 3, 2, 6, 7, 8]      13
[5, 1, 4, 6, 3, 2, 7, 8]     -38
    \end{Verbatim}

    The first better \(\Delta x\) is \(-38\). Let's move the solution to the
neighbor that applies this improvement:
\(T \leftarrow (5, 1, 4, 6, 3, 2, 7, 8)\).

    \begin{tcolorbox}[breakable, size=fbox, boxrule=1pt, pad at break*=1mm,colback=cellbackground, colframe=cellborder]
\prompt{In}{incolor}{31}{\boxspacing}
\begin{Verbatim}[commandchars=\\\{\}]
\PY{n}{tour} \PY{o}{=} \PY{p}{[}\PY{l+m+mi}{5}\PY{p}{,} \PY{l+m+mi}{1}\PY{p}{,} \PY{l+m+mi}{4}\PY{p}{,} \PY{l+m+mi}{6}\PY{p}{,} \PY{l+m+mi}{3}\PY{p}{,} \PY{l+m+mi}{2}\PY{p}{,} \PY{l+m+mi}{7}\PY{p}{,} \PY{l+m+mi}{8}\PY{p}{]}
\PY{n}{plot\PYZus{}tour}\PY{p}{(}\PY{p}{)}
\end{Verbatim}
\end{tcolorbox}

    \begin{center}
    \adjustimage{max size={0.9\linewidth}{0.9\paperheight}}{homework4_files/homework4_72_0.pdf}
    \end{center}
    { \hspace*{\fill} \\}
    
    Edges \((4,2)\) and \((6,7)\) were swapped this time.

    \hypertarget{sixth-iteration}{%
\subsubsection{Sixth iteration}\label{sixth-iteration}}

The last movement got the same solution as the last iteration of BFS.
Let's check whether or not the FFS algorithm will finish there too:

    \begin{tcolorbox}[breakable, size=fbox, boxrule=1pt, pad at break*=1mm,colback=cellbackground, colframe=cellborder]
\prompt{In}{incolor}{32}{\boxspacing}
\begin{Verbatim}[commandchars=\\\{\}]
\PY{n}{ffs}\PY{p}{(}\PY{p}{)}
\end{Verbatim}
\end{tcolorbox}

    \begin{Verbatim}[commandchars=\\\{\}]
[1, 5, 4, 6, 3, 2, 7, 8]      35
[4, 1, 5, 6, 3, 2, 7, 8]      64
[6, 4, 1, 5, 3, 2, 7, 8]     107
[3, 6, 4, 1, 5, 2, 7, 8]      91
[2, 3, 6, 4, 1, 5, 7, 8]      21
[7, 2, 3, 6, 4, 1, 5, 8]       0
[5, 4, 1, 6, 3, 2, 7, 8]      45
[5, 6, 4, 1, 3, 2, 7, 8]      99
[5, 3, 6, 4, 1, 2, 7, 8]      99
[5, 2, 3, 6, 4, 1, 7, 8]      46
[5, 7, 2, 3, 6, 4, 1, 8]      31
[5, 1, 6, 4, 3, 2, 7, 8]      66
[5, 1, 3, 6, 4, 2, 7, 8]      76
[5, 1, 2, 3, 6, 4, 7, 8]      56
[5, 1, 7, 2, 3, 6, 4, 8]      61
[5, 1, 4, 3, 6, 2, 7, 8]      24
[5, 1, 4, 2, 3, 6, 7, 8]      38
[5, 1, 4, 7, 2, 3, 6, 8]      71
[5, 1, 4, 6, 2, 3, 7, 8]      41
[5, 1, 4, 6, 7, 2, 3, 8]     103
[5, 1, 4, 6, 3, 7, 2, 8]      77
    \end{Verbatim}

    The whole neighborhood was explored without any improvement. Both
strategies come to an end with the same solution,
\(T \leftarrow (5, 1, 4, 6, 3, 2, 7, 8)\), which has a total travel cost
of \(235\).

    \hypertarget{conclusion}{%
\section{Conclusion}\label{conclusion}}

Both strategies give the same local optimal solution in this instance.
The FFS took one more iteration than the BFS, but the latter performed
more calculations of \(\Delta x\). Very big instances may take too much
time and effort using BFS, but on the other hand, using FFS may take too
many iterations. The decision of which strategy to use can be taken by
running some tests to have statistical evidence of which one is better
for a given problem or for a specific instance.


    % Add a bibliography block to the postdoc
    
    
    
\end{document}
