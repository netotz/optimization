\documentclass[11pt]{article}

    \usepackage[breakable]{tcolorbox}
    \usepackage{parskip} % Stop auto-indenting (to mimic markdown behaviour)
    
    \usepackage{iftex}
    \ifPDFTeX
    	\usepackage[T1]{fontenc}
    	\usepackage{mathpazo}
    \else
    	\usepackage{fontspec}
    \fi

    % Basic figure setup, for now with no caption control since it's done
    % automatically by Pandoc (which extracts ![](path) syntax from Markdown).
    \usepackage{graphicx}
    % Maintain compatibility with old templates. Remove in nbconvert 6.0
    \let\Oldincludegraphics\includegraphics
    % Ensure that by default, figures have no caption (until we provide a
    % proper Figure object with a Caption API and a way to capture that
    % in the conversion process - todo).
    \usepackage{caption}
    \DeclareCaptionFormat{nocaption}{}
    \captionsetup{format=nocaption,aboveskip=0pt,belowskip=0pt}

    \usepackage[Export]{adjustbox} % Used to constrain images to a maximum size
    \adjustboxset{max size={0.9\linewidth}{0.9\paperheight}}
    \usepackage{float}
    \floatplacement{figure}{H} % forces figures to be placed at the correct location
    \usepackage{xcolor} % Allow colors to be defined
    \usepackage{enumerate} % Needed for markdown enumerations to work
    \usepackage{geometry} % Used to adjust the document margins
    \usepackage{amsmath} % Equations
    \usepackage{amssymb} % Equations
    \usepackage{textcomp} % defines textquotesingle
    % Hack from http://tex.stackexchange.com/a/47451/13684:
    \AtBeginDocument{%
        \def\PYZsq{\textquotesingle}% Upright quotes in Pygmentized code
    }
    \usepackage{upquote} % Upright quotes for verbatim code
    \usepackage{eurosym} % defines \euro
    \usepackage[mathletters]{ucs} % Extended unicode (utf-8) support
    \usepackage{fancyvrb} % verbatim replacement that allows latex
    \usepackage{grffile} % extends the file name processing of package graphics 
                         % to support a larger range
    \makeatletter % fix for grffile with XeLaTeX
    \def\Gread@@xetex#1{%
      \IfFileExists{"\Gin@base".bb}%
      {\Gread@eps{\Gin@base.bb}}%
      {\Gread@@xetex@aux#1}%
    }
    \makeatother

    % The hyperref package gives us a pdf with properly built
    % internal navigation ('pdf bookmarks' for the table of contents,
    % internal cross-reference links, web links for URLs, etc.)
    \usepackage{hyperref}
    % The default LaTeX title has an obnoxious amount of whitespace. By default,
    % titling removes some of it. It also provides customization options.
    \usepackage{titling}
    \usepackage{longtable} % longtable support required by pandoc >1.10
    \usepackage{booktabs}  % table support for pandoc > 1.12.2
    \usepackage[inline]{enumitem} % IRkernel/repr support (it uses the enumerate* environment)
    \usepackage[normalem]{ulem} % ulem is needed to support strikethroughs (\sout)
                                % normalem makes italics be italics, not underlines
    \usepackage{mathrsfs}
    

    
    % Colors for the hyperref package
    \definecolor{urlcolor}{rgb}{0,.145,.698}
    \definecolor{linkcolor}{rgb}{.71,0.21,0.01}
    \definecolor{citecolor}{rgb}{.12,.54,.11}

    % ANSI colors
    \definecolor{ansi-black}{HTML}{3E424D}
    \definecolor{ansi-black-intense}{HTML}{282C36}
    \definecolor{ansi-red}{HTML}{E75C58}
    \definecolor{ansi-red-intense}{HTML}{B22B31}
    \definecolor{ansi-green}{HTML}{00A250}
    \definecolor{ansi-green-intense}{HTML}{007427}
    \definecolor{ansi-yellow}{HTML}{DDB62B}
    \definecolor{ansi-yellow-intense}{HTML}{B27D12}
    \definecolor{ansi-blue}{HTML}{208FFB}
    \definecolor{ansi-blue-intense}{HTML}{0065CA}
    \definecolor{ansi-magenta}{HTML}{D160C4}
    \definecolor{ansi-magenta-intense}{HTML}{A03196}
    \definecolor{ansi-cyan}{HTML}{60C6C8}
    \definecolor{ansi-cyan-intense}{HTML}{258F8F}
    \definecolor{ansi-white}{HTML}{C5C1B4}
    \definecolor{ansi-white-intense}{HTML}{A1A6B2}
    \definecolor{ansi-default-inverse-fg}{HTML}{FFFFFF}
    \definecolor{ansi-default-inverse-bg}{HTML}{000000}

    % commands and environments needed by pandoc snippets
    % extracted from the output of `pandoc -s`
    \providecommand{\tightlist}{%
      \setlength{\itemsep}{0pt}\setlength{\parskip}{0pt}}
    \DefineVerbatimEnvironment{Highlighting}{Verbatim}{commandchars=\\\{\}}
    % Add ',fontsize=\small' for more characters per line
    \newenvironment{Shaded}{}{}
    \newcommand{\KeywordTok}[1]{\textcolor[rgb]{0.00,0.44,0.13}{\textbf{{#1}}}}
    \newcommand{\DataTypeTok}[1]{\textcolor[rgb]{0.56,0.13,0.00}{{#1}}}
    \newcommand{\DecValTok}[1]{\textcolor[rgb]{0.25,0.63,0.44}{{#1}}}
    \newcommand{\BaseNTok}[1]{\textcolor[rgb]{0.25,0.63,0.44}{{#1}}}
    \newcommand{\FloatTok}[1]{\textcolor[rgb]{0.25,0.63,0.44}{{#1}}}
    \newcommand{\CharTok}[1]{\textcolor[rgb]{0.25,0.44,0.63}{{#1}}}
    \newcommand{\StringTok}[1]{\textcolor[rgb]{0.25,0.44,0.63}{{#1}}}
    \newcommand{\CommentTok}[1]{\textcolor[rgb]{0.38,0.63,0.69}{\textit{{#1}}}}
    \newcommand{\OtherTok}[1]{\textcolor[rgb]{0.00,0.44,0.13}{{#1}}}
    \newcommand{\AlertTok}[1]{\textcolor[rgb]{1.00,0.00,0.00}{\textbf{{#1}}}}
    \newcommand{\FunctionTok}[1]{\textcolor[rgb]{0.02,0.16,0.49}{{#1}}}
    \newcommand{\RegionMarkerTok}[1]{{#1}}
    \newcommand{\ErrorTok}[1]{\textcolor[rgb]{1.00,0.00,0.00}{\textbf{{#1}}}}
    \newcommand{\NormalTok}[1]{{#1}}
    
    % Additional commands for more recent versions of Pandoc
    \newcommand{\ConstantTok}[1]{\textcolor[rgb]{0.53,0.00,0.00}{{#1}}}
    \newcommand{\SpecialCharTok}[1]{\textcolor[rgb]{0.25,0.44,0.63}{{#1}}}
    \newcommand{\VerbatimStringTok}[1]{\textcolor[rgb]{0.25,0.44,0.63}{{#1}}}
    \newcommand{\SpecialStringTok}[1]{\textcolor[rgb]{0.73,0.40,0.53}{{#1}}}
    \newcommand{\ImportTok}[1]{{#1}}
    \newcommand{\DocumentationTok}[1]{\textcolor[rgb]{0.73,0.13,0.13}{\textit{{#1}}}}
    \newcommand{\AnnotationTok}[1]{\textcolor[rgb]{0.38,0.63,0.69}{\textbf{\textit{{#1}}}}}
    \newcommand{\CommentVarTok}[1]{\textcolor[rgb]{0.38,0.63,0.69}{\textbf{\textit{{#1}}}}}
    \newcommand{\VariableTok}[1]{\textcolor[rgb]{0.10,0.09,0.49}{{#1}}}
    \newcommand{\ControlFlowTok}[1]{\textcolor[rgb]{0.00,0.44,0.13}{\textbf{{#1}}}}
    \newcommand{\OperatorTok}[1]{\textcolor[rgb]{0.40,0.40,0.40}{{#1}}}
    \newcommand{\BuiltInTok}[1]{{#1}}
    \newcommand{\ExtensionTok}[1]{{#1}}
    \newcommand{\PreprocessorTok}[1]{\textcolor[rgb]{0.74,0.48,0.00}{{#1}}}
    \newcommand{\AttributeTok}[1]{\textcolor[rgb]{0.49,0.56,0.16}{{#1}}}
    \newcommand{\InformationTok}[1]{\textcolor[rgb]{0.38,0.63,0.69}{\textbf{\textit{{#1}}}}}
    \newcommand{\WarningTok}[1]{\textcolor[rgb]{0.38,0.63,0.69}{\textbf{\textit{{#1}}}}}
    
    
    % Define a nice break command that doesn't care if a line doesn't already
    % exist.
    \def\br{\hspace*{\fill} \\* }
    % Math Jax compatibility definitions
    \def\gt{>}
    \def\lt{<}
    \let\Oldtex\TeX
    \let\Oldlatex\LaTeX
    \renewcommand{\TeX}{\textrm{\Oldtex}}
    \renewcommand{\LaTeX}{\textrm{\Oldlatex}}
    % Document parameters
    % Document title 
    \title{Travelling Salesman Problem}
    
    % author
    \author{Ernesto Ortiz}
    
    
% Pygments definitions
\makeatletter
\def\PY@reset{\let\PY@it=\relax \let\PY@bf=\relax%
    \let\PY@ul=\relax \let\PY@tc=\relax%
    \let\PY@bc=\relax \let\PY@ff=\relax}
\def\PY@tok#1{\csname PY@tok@#1\endcsname}
\def\PY@toks#1+{\ifx\relax#1\empty\else%
    \PY@tok{#1}\expandafter\PY@toks\fi}
\def\PY@do#1{\PY@bc{\PY@tc{\PY@ul{%
    \PY@it{\PY@bf{\PY@ff{#1}}}}}}}
\def\PY#1#2{\PY@reset\PY@toks#1+\relax+\PY@do{#2}}

\expandafter\def\csname PY@tok@w\endcsname{\def\PY@tc##1{\textcolor[rgb]{0.73,0.73,0.73}{##1}}}
\expandafter\def\csname PY@tok@c\endcsname{\let\PY@it=\textit\def\PY@tc##1{\textcolor[rgb]{0.25,0.50,0.50}{##1}}}
\expandafter\def\csname PY@tok@cp\endcsname{\def\PY@tc##1{\textcolor[rgb]{0.74,0.48,0.00}{##1}}}
\expandafter\def\csname PY@tok@k\endcsname{\let\PY@bf=\textbf\def\PY@tc##1{\textcolor[rgb]{0.00,0.50,0.00}{##1}}}
\expandafter\def\csname PY@tok@kp\endcsname{\def\PY@tc##1{\textcolor[rgb]{0.00,0.50,0.00}{##1}}}
\expandafter\def\csname PY@tok@kt\endcsname{\def\PY@tc##1{\textcolor[rgb]{0.69,0.00,0.25}{##1}}}
\expandafter\def\csname PY@tok@o\endcsname{\def\PY@tc##1{\textcolor[rgb]{0.40,0.40,0.40}{##1}}}
\expandafter\def\csname PY@tok@ow\endcsname{\let\PY@bf=\textbf\def\PY@tc##1{\textcolor[rgb]{0.67,0.13,1.00}{##1}}}
\expandafter\def\csname PY@tok@nb\endcsname{\def\PY@tc##1{\textcolor[rgb]{0.00,0.50,0.00}{##1}}}
\expandafter\def\csname PY@tok@nf\endcsname{\def\PY@tc##1{\textcolor[rgb]{0.00,0.00,1.00}{##1}}}
\expandafter\def\csname PY@tok@nc\endcsname{\let\PY@bf=\textbf\def\PY@tc##1{\textcolor[rgb]{0.00,0.00,1.00}{##1}}}
\expandafter\def\csname PY@tok@nn\endcsname{\let\PY@bf=\textbf\def\PY@tc##1{\textcolor[rgb]{0.00,0.00,1.00}{##1}}}
\expandafter\def\csname PY@tok@ne\endcsname{\let\PY@bf=\textbf\def\PY@tc##1{\textcolor[rgb]{0.82,0.25,0.23}{##1}}}
\expandafter\def\csname PY@tok@nv\endcsname{\def\PY@tc##1{\textcolor[rgb]{0.10,0.09,0.49}{##1}}}
\expandafter\def\csname PY@tok@no\endcsname{\def\PY@tc##1{\textcolor[rgb]{0.53,0.00,0.00}{##1}}}
\expandafter\def\csname PY@tok@nl\endcsname{\def\PY@tc##1{\textcolor[rgb]{0.63,0.63,0.00}{##1}}}
\expandafter\def\csname PY@tok@ni\endcsname{\let\PY@bf=\textbf\def\PY@tc##1{\textcolor[rgb]{0.60,0.60,0.60}{##1}}}
\expandafter\def\csname PY@tok@na\endcsname{\def\PY@tc##1{\textcolor[rgb]{0.49,0.56,0.16}{##1}}}
\expandafter\def\csname PY@tok@nt\endcsname{\let\PY@bf=\textbf\def\PY@tc##1{\textcolor[rgb]{0.00,0.50,0.00}{##1}}}
\expandafter\def\csname PY@tok@nd\endcsname{\def\PY@tc##1{\textcolor[rgb]{0.67,0.13,1.00}{##1}}}
\expandafter\def\csname PY@tok@s\endcsname{\def\PY@tc##1{\textcolor[rgb]{0.73,0.13,0.13}{##1}}}
\expandafter\def\csname PY@tok@sd\endcsname{\let\PY@it=\textit\def\PY@tc##1{\textcolor[rgb]{0.73,0.13,0.13}{##1}}}
\expandafter\def\csname PY@tok@si\endcsname{\let\PY@bf=\textbf\def\PY@tc##1{\textcolor[rgb]{0.73,0.40,0.53}{##1}}}
\expandafter\def\csname PY@tok@se\endcsname{\let\PY@bf=\textbf\def\PY@tc##1{\textcolor[rgb]{0.73,0.40,0.13}{##1}}}
\expandafter\def\csname PY@tok@sr\endcsname{\def\PY@tc##1{\textcolor[rgb]{0.73,0.40,0.53}{##1}}}
\expandafter\def\csname PY@tok@ss\endcsname{\def\PY@tc##1{\textcolor[rgb]{0.10,0.09,0.49}{##1}}}
\expandafter\def\csname PY@tok@sx\endcsname{\def\PY@tc##1{\textcolor[rgb]{0.00,0.50,0.00}{##1}}}
\expandafter\def\csname PY@tok@m\endcsname{\def\PY@tc##1{\textcolor[rgb]{0.40,0.40,0.40}{##1}}}
\expandafter\def\csname PY@tok@gh\endcsname{\let\PY@bf=\textbf\def\PY@tc##1{\textcolor[rgb]{0.00,0.00,0.50}{##1}}}
\expandafter\def\csname PY@tok@gu\endcsname{\let\PY@bf=\textbf\def\PY@tc##1{\textcolor[rgb]{0.50,0.00,0.50}{##1}}}
\expandafter\def\csname PY@tok@gd\endcsname{\def\PY@tc##1{\textcolor[rgb]{0.63,0.00,0.00}{##1}}}
\expandafter\def\csname PY@tok@gi\endcsname{\def\PY@tc##1{\textcolor[rgb]{0.00,0.63,0.00}{##1}}}
\expandafter\def\csname PY@tok@gr\endcsname{\def\PY@tc##1{\textcolor[rgb]{1.00,0.00,0.00}{##1}}}
\expandafter\def\csname PY@tok@ge\endcsname{\let\PY@it=\textit}
\expandafter\def\csname PY@tok@gs\endcsname{\let\PY@bf=\textbf}
\expandafter\def\csname PY@tok@gp\endcsname{\let\PY@bf=\textbf\def\PY@tc##1{\textcolor[rgb]{0.00,0.00,0.50}{##1}}}
\expandafter\def\csname PY@tok@go\endcsname{\def\PY@tc##1{\textcolor[rgb]{0.53,0.53,0.53}{##1}}}
\expandafter\def\csname PY@tok@gt\endcsname{\def\PY@tc##1{\textcolor[rgb]{0.00,0.27,0.87}{##1}}}
\expandafter\def\csname PY@tok@err\endcsname{\def\PY@bc##1{\setlength{\fboxsep}{0pt}\fcolorbox[rgb]{1.00,0.00,0.00}{1,1,1}{\strut ##1}}}
\expandafter\def\csname PY@tok@kc\endcsname{\let\PY@bf=\textbf\def\PY@tc##1{\textcolor[rgb]{0.00,0.50,0.00}{##1}}}
\expandafter\def\csname PY@tok@kd\endcsname{\let\PY@bf=\textbf\def\PY@tc##1{\textcolor[rgb]{0.00,0.50,0.00}{##1}}}
\expandafter\def\csname PY@tok@kn\endcsname{\let\PY@bf=\textbf\def\PY@tc##1{\textcolor[rgb]{0.00,0.50,0.00}{##1}}}
\expandafter\def\csname PY@tok@kr\endcsname{\let\PY@bf=\textbf\def\PY@tc##1{\textcolor[rgb]{0.00,0.50,0.00}{##1}}}
\expandafter\def\csname PY@tok@bp\endcsname{\def\PY@tc##1{\textcolor[rgb]{0.00,0.50,0.00}{##1}}}
\expandafter\def\csname PY@tok@fm\endcsname{\def\PY@tc##1{\textcolor[rgb]{0.00,0.00,1.00}{##1}}}
\expandafter\def\csname PY@tok@vc\endcsname{\def\PY@tc##1{\textcolor[rgb]{0.10,0.09,0.49}{##1}}}
\expandafter\def\csname PY@tok@vg\endcsname{\def\PY@tc##1{\textcolor[rgb]{0.10,0.09,0.49}{##1}}}
\expandafter\def\csname PY@tok@vi\endcsname{\def\PY@tc##1{\textcolor[rgb]{0.10,0.09,0.49}{##1}}}
\expandafter\def\csname PY@tok@vm\endcsname{\def\PY@tc##1{\textcolor[rgb]{0.10,0.09,0.49}{##1}}}
\expandafter\def\csname PY@tok@sa\endcsname{\def\PY@tc##1{\textcolor[rgb]{0.73,0.13,0.13}{##1}}}
\expandafter\def\csname PY@tok@sb\endcsname{\def\PY@tc##1{\textcolor[rgb]{0.73,0.13,0.13}{##1}}}
\expandafter\def\csname PY@tok@sc\endcsname{\def\PY@tc##1{\textcolor[rgb]{0.73,0.13,0.13}{##1}}}
\expandafter\def\csname PY@tok@dl\endcsname{\def\PY@tc##1{\textcolor[rgb]{0.73,0.13,0.13}{##1}}}
\expandafter\def\csname PY@tok@s2\endcsname{\def\PY@tc##1{\textcolor[rgb]{0.73,0.13,0.13}{##1}}}
\expandafter\def\csname PY@tok@sh\endcsname{\def\PY@tc##1{\textcolor[rgb]{0.73,0.13,0.13}{##1}}}
\expandafter\def\csname PY@tok@s1\endcsname{\def\PY@tc##1{\textcolor[rgb]{0.73,0.13,0.13}{##1}}}
\expandafter\def\csname PY@tok@mb\endcsname{\def\PY@tc##1{\textcolor[rgb]{0.40,0.40,0.40}{##1}}}
\expandafter\def\csname PY@tok@mf\endcsname{\def\PY@tc##1{\textcolor[rgb]{0.40,0.40,0.40}{##1}}}
\expandafter\def\csname PY@tok@mh\endcsname{\def\PY@tc##1{\textcolor[rgb]{0.40,0.40,0.40}{##1}}}
\expandafter\def\csname PY@tok@mi\endcsname{\def\PY@tc##1{\textcolor[rgb]{0.40,0.40,0.40}{##1}}}
\expandafter\def\csname PY@tok@il\endcsname{\def\PY@tc##1{\textcolor[rgb]{0.40,0.40,0.40}{##1}}}
\expandafter\def\csname PY@tok@mo\endcsname{\def\PY@tc##1{\textcolor[rgb]{0.40,0.40,0.40}{##1}}}
\expandafter\def\csname PY@tok@ch\endcsname{\let\PY@it=\textit\def\PY@tc##1{\textcolor[rgb]{0.25,0.50,0.50}{##1}}}
\expandafter\def\csname PY@tok@cm\endcsname{\let\PY@it=\textit\def\PY@tc##1{\textcolor[rgb]{0.25,0.50,0.50}{##1}}}
\expandafter\def\csname PY@tok@cpf\endcsname{\let\PY@it=\textit\def\PY@tc##1{\textcolor[rgb]{0.25,0.50,0.50}{##1}}}
\expandafter\def\csname PY@tok@c1\endcsname{\let\PY@it=\textit\def\PY@tc##1{\textcolor[rgb]{0.25,0.50,0.50}{##1}}}
\expandafter\def\csname PY@tok@cs\endcsname{\let\PY@it=\textit\def\PY@tc##1{\textcolor[rgb]{0.25,0.50,0.50}{##1}}}

\def\PYZbs{\char`\\}
\def\PYZus{\char`\_}
\def\PYZob{\char`\{}
\def\PYZcb{\char`\}}
\def\PYZca{\char`\^}
\def\PYZam{\char`\&}
\def\PYZlt{\char`\<}
\def\PYZgt{\char`\>}
\def\PYZsh{\char`\#}
\def\PYZpc{\char`\%}
\def\PYZdl{\char`\$}
\def\PYZhy{\char`\-}
\def\PYZsq{\char`\'}
\def\PYZdq{\char`\"}
\def\PYZti{\char`\~}
% for compatibility with earlier versions
\def\PYZat{@}
\def\PYZlb{[}
\def\PYZrb{]}
\makeatother


    % For linebreaks inside Verbatim environment from package fancyvrb. 
    \makeatletter
        \newbox\Wrappedcontinuationbox 
        \newbox\Wrappedvisiblespacebox 
        \newcommand*\Wrappedvisiblespace {\textcolor{red}{\textvisiblespace}} 
        \newcommand*\Wrappedcontinuationsymbol {\textcolor{red}{\llap{\tiny$\m@th\hookrightarrow$}}} 
        \newcommand*\Wrappedcontinuationindent {3ex } 
        \newcommand*\Wrappedafterbreak {\kern\Wrappedcontinuationindent\copy\Wrappedcontinuationbox} 
        % Take advantage of the already applied Pygments mark-up to insert 
        % potential linebreaks for TeX processing. 
        %        {, <, #, %, $, ' and ": go to next line. 
        %        _, }, ^, &, >, - and ~: stay at end of broken line. 
        % Use of \textquotesingle for straight quote. 
        \newcommand*\Wrappedbreaksatspecials {% 
            \def\PYGZus{\discretionary{\char`\_}{\Wrappedafterbreak}{\char`\_}}% 
            \def\PYGZob{\discretionary{}{\Wrappedafterbreak\char`\{}{\char`\{}}% 
            \def\PYGZcb{\discretionary{\char`\}}{\Wrappedafterbreak}{\char`\}}}% 
            \def\PYGZca{\discretionary{\char`\^}{\Wrappedafterbreak}{\char`\^}}% 
            \def\PYGZam{\discretionary{\char`\&}{\Wrappedafterbreak}{\char`\&}}% 
            \def\PYGZlt{\discretionary{}{\Wrappedafterbreak\char`\<}{\char`\<}}% 
            \def\PYGZgt{\discretionary{\char`\>}{\Wrappedafterbreak}{\char`\>}}% 
            \def\PYGZsh{\discretionary{}{\Wrappedafterbreak\char`\#}{\char`\#}}% 
            \def\PYGZpc{\discretionary{}{\Wrappedafterbreak\char`\%}{\char`\%}}% 
            \def\PYGZdl{\discretionary{}{\Wrappedafterbreak\char`\$}{\char`\$}}% 
            \def\PYGZhy{\discretionary{\char`\-}{\Wrappedafterbreak}{\char`\-}}% 
            \def\PYGZsq{\discretionary{}{\Wrappedafterbreak\textquotesingle}{\textquotesingle}}% 
            \def\PYGZdq{\discretionary{}{\Wrappedafterbreak\char`\"}{\char`\"}}% 
            \def\PYGZti{\discretionary{\char`\~}{\Wrappedafterbreak}{\char`\~}}% 
        } 
        % Some characters . , ; ? ! / are not pygmentized. 
        % This macro makes them "active" and they will insert potential linebreaks 
        \newcommand*\Wrappedbreaksatpunct {% 
            \lccode`\~`\.\lowercase{\def~}{\discretionary{\hbox{\char`\.}}{\Wrappedafterbreak}{\hbox{\char`\.}}}% 
            \lccode`\~`\,\lowercase{\def~}{\discretionary{\hbox{\char`\,}}{\Wrappedafterbreak}{\hbox{\char`\,}}}% 
            \lccode`\~`\;\lowercase{\def~}{\discretionary{\hbox{\char`\;}}{\Wrappedafterbreak}{\hbox{\char`\;}}}% 
            \lccode`\~`\:\lowercase{\def~}{\discretionary{\hbox{\char`\:}}{\Wrappedafterbreak}{\hbox{\char`\:}}}% 
            \lccode`\~`\?\lowercase{\def~}{\discretionary{\hbox{\char`\?}}{\Wrappedafterbreak}{\hbox{\char`\?}}}% 
            \lccode`\~`\!\lowercase{\def~}{\discretionary{\hbox{\char`\!}}{\Wrappedafterbreak}{\hbox{\char`\!}}}% 
            \lccode`\~`\/\lowercase{\def~}{\discretionary{\hbox{\char`\/}}{\Wrappedafterbreak}{\hbox{\char`\/}}}% 
            \catcode`\.\active
            \catcode`\,\active 
            \catcode`\;\active
            \catcode`\:\active
            \catcode`\?\active
            \catcode`\!\active
            \catcode`\/\active 
            \lccode`\~`\~ 	
        }
    \makeatother

    \let\OriginalVerbatim=\Verbatim
    \makeatletter
    \renewcommand{\Verbatim}[1][1]{%
        %\parskip\z@skip
        \sbox\Wrappedcontinuationbox {\Wrappedcontinuationsymbol}%
        \sbox\Wrappedvisiblespacebox {\FV@SetupFont\Wrappedvisiblespace}%
        \def\FancyVerbFormatLine ##1{\hsize\linewidth
            \vtop{\raggedright\hyphenpenalty\z@\exhyphenpenalty\z@
                \doublehyphendemerits\z@\finalhyphendemerits\z@
                \strut ##1\strut}%
        }%
        % If the linebreak is at a space, the latter will be displayed as visible
        % space at end of first line, and a continuation symbol starts next line.
        % Stretch/shrink are however usually zero for typewriter font.
        \def\FV@Space {%
            \nobreak\hskip\z@ plus\fontdimen3\font minus\fontdimen4\font
            \discretionary{\copy\Wrappedvisiblespacebox}{\Wrappedafterbreak}
            {\kern\fontdimen2\font}%
        }%
        
        % Allow breaks at special characters using \PYG... macros.
        \Wrappedbreaksatspecials
        % Breaks at punctuation characters . , ; ? ! and / need catcode=\active 	
        \OriginalVerbatim[#1,codes*=\Wrappedbreaksatpunct]%
    }
    \makeatother

    % Exact colors from NB
    \definecolor{incolor}{HTML}{303F9F}
    \definecolor{outcolor}{HTML}{D84315}
    \definecolor{cellborder}{HTML}{CFCFCF}
    \definecolor{cellbackground}{HTML}{F7F7F7}
    
    % prompt
    \makeatletter
    \newcommand{\boxspacing}{\kern\kvtcb@left@rule\kern\kvtcb@boxsep}
    \makeatother
    \newcommand{\prompt}[4]{
        \ttfamily\llap{{\color{#2}[#3]:\hspace{3pt}#4}}\vspace{-\baselineskip}
    }
    

    
    % Prevent overflowing lines due to hard-to-break entities
    \sloppy 
    % Setup hyperref package
    \hypersetup{
      breaklinks=true,  % so long urls are correctly broken across lines
      colorlinks=true,
      urlcolor=urlcolor,
      linkcolor=linkcolor,
      citecolor=citecolor,
      }
    % Slightly bigger margins than the latex defaults
    
    \geometry{verbose,tmargin=1in,bmargin=1in,lmargin=1in,rmargin=1in}
    
    

\begin{document}
    
    \maketitle
    
    

    
    \hypertarget{data}{%
\section{Data}\label{data}}

\begin{itemize}
\item
  \(C=\{1,2,3,4,5,6,7,8\}\) \(\leftarrow\) set of cities to visit.
\item
  \(n = 8\) \(\leftarrow\) number of cities.
\end{itemize}

    \begin{tcolorbox}[breakable, size=fbox, boxrule=1pt, pad at break*=1mm,colback=cellbackground, colframe=cellborder]
\prompt{In}{incolor}{1}{\boxspacing}
\begin{Verbatim}[commandchars=\\\{\}]
\PY{n}{cities} \PY{o}{=} \PY{p}{[}\PY{n}{c} \PY{o}{+} \PY{l+m+mi}{1} \PY{k}{for} \PY{n}{c} \PY{o+ow}{in} \PY{n+nb}{range}\PY{p}{(}\PY{l+m+mi}{8}\PY{p}{)}\PY{p}{]}
\PY{n}{cities}
\end{Verbatim}
\end{tcolorbox}

            \begin{tcolorbox}[breakable, size=fbox, boxrule=.5pt, pad at break*=1mm, opacityfill=0]
\prompt{Out}{outcolor}{1}{\boxspacing}
\begin{Verbatim}[commandchars=\\\{\}]
[1, 2, 3, 4, 5, 6, 7, 8]
\end{Verbatim}
\end{tcolorbox}
        
    \begin{itemize}
\tightlist
\item
  \(D=(d_{ij})\), \(i,j \in C\) \(\leftarrow\) matrix of distances
  between cities.
\end{itemize}

    \begin{tcolorbox}[breakable, size=fbox, boxrule=1pt, pad at break*=1mm,colback=cellbackground, colframe=cellborder]
\prompt{In}{incolor}{2}{\boxspacing}
\begin{Verbatim}[commandchars=\\\{\}]
\PY{n}{distances} \PY{o}{=} \PY{p}{[}
    \PY{p}{[}\PY{l+m+mi}{0}\PY{p}{,} \PY{l+m+mi}{89}\PY{p}{,} \PY{l+m+mi}{87}\PY{p}{,} \PY{l+m+mi}{38}\PY{p}{,} \PY{l+m+mi}{33}\PY{p}{,} \PY{l+m+mi}{71}\PY{p}{,} \PY{l+m+mi}{59}\PY{p}{,} \PY{l+m+mi}{54}\PY{p}{]}\PY{p}{,}
    \PY{p}{[}\PY{l+m+mi}{89}\PY{p}{,} \PY{l+m+mi}{0}\PY{p}{,} \PY{l+m+mi}{32}\PY{p}{,} \PY{l+m+mi}{59}\PY{p}{,} \PY{l+m+mi}{65}\PY{p}{,} \PY{l+m+mi}{39}\PY{p}{,} \PY{l+m+mi}{45}\PY{p}{,} \PY{l+m+mi}{61}\PY{p}{]}\PY{p}{,}
    \PY{p}{[}\PY{l+m+mi}{87}\PY{p}{,} \PY{l+m+mi}{32}\PY{p}{,} \PY{l+m+mi}{0}\PY{p}{,} \PY{l+m+mi}{50}\PY{p}{,} \PY{l+m+mi}{75}\PY{p}{,} \PY{l+m+mi}{17}\PY{p}{,} \PY{l+m+mi}{64}\PY{p}{,} \PY{l+m+mi}{79}\PY{p}{]}\PY{p}{,}
    \PY{p}{[}\PY{l+m+mi}{38}\PY{p}{,} \PY{l+m+mi}{59}\PY{p}{,} \PY{l+m+mi}{50}\PY{p}{,} \PY{l+m+mi}{0}\PY{p}{,} \PY{l+m+mi}{40}\PY{p}{,} \PY{l+m+mi}{33}\PY{p}{,} \PY{l+m+mi}{50}\PY{p}{,} \PY{l+m+mi}{56}\PY{p}{]}\PY{p}{,}
    \PY{p}{[}\PY{l+m+mi}{33}\PY{p}{,} \PY{l+m+mi}{65}\PY{p}{,} \PY{l+m+mi}{75}\PY{p}{,} \PY{l+m+mi}{40}\PY{p}{,} \PY{l+m+mi}{0}\PY{p}{,} \PY{l+m+mi}{62}\PY{p}{,} \PY{l+m+mi}{26}\PY{p}{,} \PY{l+m+mi}{21}\PY{p}{]}\PY{p}{,}
    \PY{p}{[}\PY{l+m+mi}{71}\PY{p}{,} \PY{l+m+mi}{39}\PY{p}{,} \PY{l+m+mi}{17}\PY{p}{,} \PY{l+m+mi}{33}\PY{p}{,} \PY{l+m+mi}{62}\PY{p}{,} \PY{l+m+mi}{0}\PY{p}{,} \PY{l+m+mi}{57}\PY{p}{,} \PY{l+m+mi}{70}\PY{p}{]}\PY{p}{,}
    \PY{p}{[}\PY{l+m+mi}{59}\PY{p}{,} \PY{l+m+mi}{45}\PY{p}{,} \PY{l+m+mi}{64}\PY{p}{,} \PY{l+m+mi}{50}\PY{p}{,} \PY{l+m+mi}{26}\PY{p}{,} \PY{l+m+mi}{57}\PY{p}{,} \PY{l+m+mi}{0}\PY{p}{,} \PY{l+m+mi}{16}\PY{p}{]}\PY{p}{,}
    \PY{p}{[}\PY{l+m+mi}{54}\PY{p}{,} \PY{l+m+mi}{61}\PY{p}{,} \PY{l+m+mi}{79}\PY{p}{,} \PY{l+m+mi}{56}\PY{p}{,} \PY{l+m+mi}{21}\PY{p}{,} \PY{l+m+mi}{70}\PY{p}{,} \PY{l+m+mi}{16}\PY{p}{,} \PY{l+m+mi}{0}\PY{p}{]}
\PY{p}{]}
\end{Verbatim}
\end{tcolorbox}

    \hypertarget{decision}{%
\section{Decision}\label{decision}}

\begin{itemize}
\item
  \(T=\{t_1,t_2,t_3,t_4,t_5,t_6,t_7,t_8,t_9=t_1\}\), \(t_i \in C\)
  \(\leftarrow\) ordered set of visited cities.
\item
  \(T = \{\}\) \(\leftarrow\) initialize tour as an empty set.
\end{itemize}

    \begin{tcolorbox}[breakable, size=fbox, boxrule=1pt, pad at break*=1mm,colback=cellbackground, colframe=cellborder]
\prompt{In}{incolor}{3}{\boxspacing}
\begin{Verbatim}[commandchars=\\\{\}]
\PY{n}{tour} \PY{o}{=} \PY{p}{[}\PY{p}{]}
\end{Verbatim}
\end{tcolorbox}

    \hypertarget{objective}{%
\section{Objective}\label{objective}}

The objective is to get a tour \(T\) that visits all cities in \(C\),
ending with the first city, such that the total distance of the tour is
as minimal as possible.

    \[
\begin{aligned}
& \underset{t \in T}{\text{min}}
& & \sum_{i=1}^{n} d_{t_i t_{i+1}}
\end{aligned}
\]

    \hypertarget{procedure}{%
\section{Procedure}\label{procedure}}

    \hypertarget{initial-step}{%
\subsection{Initial step}\label{initial-step}}

To start the tour, the two nearest cities will be chosen first; that is,
we will find the smallest edge in the matrix of distances, excluding the
positions of the same cities:

    \begin{tcolorbox}[breakable, size=fbox, boxrule=1pt, pad at break*=1mm,colback=cellbackground, colframe=cellborder]
\prompt{In}{incolor}{4}{\boxspacing}
\begin{Verbatim}[commandchars=\\\{\}]
\PY{n}{minimums} \PY{o}{=} \PY{p}{(}\PY{n+nb}{min}\PY{p}{(}\PY{n+nb}{enumerate}\PY{p}{(}\PY{n}{row}\PY{p}{)}\PY{p}{,} \PY{n}{key}\PY{o}{=}\PY{k}{lambda} \PY{n}{l}\PY{p}{:} \PY{n}{l}\PY{p}{[}\PY{l+m+mi}{1}\PY{p}{]} \PY{k}{if} \PY{n}{l}\PY{p}{[}\PY{l+m+mi}{1}\PY{p}{]} \PY{o}{\PYZgt{}} \PY{l+m+mi}{0} \PY{k}{else} \PY{n+nb}{float}\PY{p}{(}\PY{l+s+s1}{\PYZsq{}}\PY{l+s+s1}{inf}\PY{l+s+s1}{\PYZsq{}}\PY{p}{)}\PY{p}{)} \PY{k}{for} \PY{n}{\PYZus{}}\PY{p}{,} \PY{n}{row} \PY{o+ow}{in} \PY{n+nb}{enumerate}\PY{p}{(}\PY{n}{distances}\PY{p}{)}\PY{p}{)}
\PY{n}{smallest} \PY{o}{=} \PY{n+nb}{min}\PY{p}{(}\PY{n+nb}{enumerate}\PY{p}{(}\PY{n}{minimums}\PY{p}{)}\PY{p}{,} \PY{n}{key}\PY{o}{=}\PY{k}{lambda} \PY{n}{m}\PY{p}{:} \PY{n}{m}\PY{p}{[}\PY{l+m+mi}{1}\PY{p}{]}\PY{p}{[}\PY{l+m+mi}{1}\PY{p}{]}\PY{p}{)}
\PY{n}{smallest}\PY{p}{[}\PY{l+m+mi}{1}\PY{p}{]}\PY{p}{[}\PY{l+m+mi}{1}\PY{p}{]}
\end{Verbatim}
\end{tcolorbox}

            \begin{tcolorbox}[breakable, size=fbox, boxrule=.5pt, pad at break*=1mm, opacityfill=0]
\prompt{Out}{outcolor}{4}{\boxspacing}
\begin{Verbatim}[commandchars=\\\{\}]
16
\end{Verbatim}
\end{tcolorbox}
        
    The smallest edge is \(16\), but what we need are the coordinates (the
cities) to which that edge belongs:

    \begin{tcolorbox}[breakable, size=fbox, boxrule=1pt, pad at break*=1mm,colback=cellbackground, colframe=cellborder]
\prompt{In}{incolor}{5}{\boxspacing}
\begin{Verbatim}[commandchars=\\\{\}]
\PY{n}{c1} \PY{o}{=} \PY{n}{smallest}\PY{p}{[}\PY{l+m+mi}{0}\PY{p}{]} \PY{o}{+} \PY{l+m+mi}{1}
\PY{n}{c2} \PY{o}{=} \PY{n}{smallest}\PY{p}{[}\PY{l+m+mi}{1}\PY{p}{]}\PY{p}{[}\PY{l+m+mi}{0}\PY{p}{]} \PY{o}{+} \PY{l+m+mi}{1}
\PY{n}{c1}\PY{p}{,} \PY{n}{c2}
\end{Verbatim}
\end{tcolorbox}

            \begin{tcolorbox}[breakable, size=fbox, boxrule=.5pt, pad at break*=1mm, opacityfill=0]
\prompt{Out}{outcolor}{5}{\boxspacing}
\begin{Verbatim}[commandchars=\\\{\}]
(7, 8)
\end{Verbatim}
\end{tcolorbox}
        
    Now we can append cities \(7\) and \(8\) to our tour \(T\):

\(T=T\oplus\{7, 8\}\)

    \begin{tcolorbox}[breakable, size=fbox, boxrule=1pt, pad at break*=1mm,colback=cellbackground, colframe=cellborder]
\prompt{In}{incolor}{6}{\boxspacing}
\begin{Verbatim}[commandchars=\\\{\}]
\PY{n}{tour}\PY{o}{.}\PY{n}{extend}\PY{p}{(}\PY{p}{[}\PY{l+m+mi}{7}\PY{p}{,} \PY{l+m+mi}{8}\PY{p}{]}\PY{p}{)}
\PY{n}{tour}
\end{Verbatim}
\end{tcolorbox}

            \begin{tcolorbox}[breakable, size=fbox, boxrule=.5pt, pad at break*=1mm, opacityfill=0]
\prompt{Out}{outcolor}{6}{\boxspacing}
\begin{Verbatim}[commandchars=\\\{\}]
[7, 8]
\end{Verbatim}
\end{tcolorbox}
        
    In the TSP, the tour ends with the first visited city, so our tour \(T\)
should look like \(\{7, 8, \ldots, 7\}\):

    \begin{tcolorbox}[breakable, size=fbox, boxrule=1pt, pad at break*=1mm,colback=cellbackground, colframe=cellborder]
\prompt{In}{incolor}{7}{\boxspacing}
\begin{Verbatim}[commandchars=\\\{\}]
\PY{n}{tour}\PY{o}{.}\PY{n}{append}\PY{p}{(}\PY{l+m+mi}{7}\PY{p}{)}
\PY{n}{tour}
\end{Verbatim}
\end{tcolorbox}

            \begin{tcolorbox}[breakable, size=fbox, boxrule=.5pt, pad at break*=1mm, opacityfill=0]
\prompt{Out}{outcolor}{7}{\boxspacing}
\begin{Verbatim}[commandchars=\\\{\}]
[7, 8, 7]
\end{Verbatim}
\end{tcolorbox}
        
    The previous selected cities need to be removed from the set of cities
\(C\). In order to avoid modifying \(C\), we will use a new set to store
the unvisited cities:

\(\bar{C}=C\setminus\{7, 8\}\)

    \begin{tcolorbox}[breakable, size=fbox, boxrule=1pt, pad at break*=1mm,colback=cellbackground, colframe=cellborder]
\prompt{In}{incolor}{8}{\boxspacing}
\begin{Verbatim}[commandchars=\\\{\}]
\PY{n}{cities\PYZus{}bar} \PY{o}{=} \PY{n+nb}{list}\PY{p}{(}\PY{n}{cities}\PY{p}{)}
\PY{n}{cities\PYZus{}bar}\PY{o}{.}\PY{n}{remove}\PY{p}{(}\PY{l+m+mi}{7}\PY{p}{)}
\PY{n}{cities\PYZus{}bar}\PY{o}{.}\PY{n}{remove}\PY{p}{(}\PY{l+m+mi}{8}\PY{p}{)}
\PY{n}{cities\PYZus{}bar}
\end{Verbatim}
\end{tcolorbox}

            \begin{tcolorbox}[breakable, size=fbox, boxrule=.5pt, pad at break*=1mm, opacityfill=0]
\prompt{Out}{outcolor}{8}{\boxspacing}
\begin{Verbatim}[commandchars=\\\{\}]
[1, 2, 3, 4, 5, 6]
\end{Verbatim}
\end{tcolorbox}
        
    \hypertarget{plotting}{%
\subsection{Plotting}\label{plotting}}

To illustrate the path of each subtour, we will declare a function to
make the corresponding plot for every iteration. Before anything else,
let's declare the coordinates of the cities:

    \begin{tcolorbox}[breakable, size=fbox, boxrule=1pt, pad at break*=1mm,colback=cellbackground, colframe=cellborder]
\prompt{In}{incolor}{9}{\boxspacing}
\begin{Verbatim}[commandchars=\\\{\}]
\PY{n}{coords} \PY{o}{=} \PY{p}{(}    \PY{c+c1}{\PYZsh{} cities}
    \PY{p}{(}\PY{l+m+mi}{86}\PY{p}{,} \PY{l+m+mi}{37}\PY{p}{)}\PY{p}{,} \PY{c+c1}{\PYZsh{} 1}
    \PY{p}{(}\PY{l+m+mi}{17}\PY{p}{,} \PY{l+m+mi}{94}\PY{p}{)}\PY{p}{,} \PY{c+c1}{\PYZsh{} 2}
    \PY{p}{(}\PY{l+m+mi}{3}\PY{p}{,} \PY{l+m+mi}{65}\PY{p}{)}\PY{p}{,}  \PY{c+c1}{\PYZsh{} 3}
    \PY{p}{(}\PY{l+m+mi}{48}\PY{p}{,} \PY{l+m+mi}{43}\PY{p}{)}\PY{p}{,} \PY{c+c1}{\PYZsh{} 4}
    \PY{p}{(}\PY{l+m+mi}{78}\PY{p}{,} \PY{l+m+mi}{70}\PY{p}{)}\PY{p}{,} \PY{c+c1}{\PYZsh{} 5}
    \PY{p}{(}\PY{l+m+mi}{17}\PY{p}{,} \PY{l+m+mi}{55}\PY{p}{)}\PY{p}{,} \PY{c+c1}{\PYZsh{} 6}
    \PY{p}{(}\PY{l+m+mi}{62}\PY{p}{,} \PY{l+m+mi}{91}\PY{p}{)}\PY{p}{,} \PY{c+c1}{\PYZsh{} 7}
    \PY{p}{(}\PY{l+m+mi}{78}\PY{p}{,} \PY{l+m+mi}{91}\PY{p}{)}  \PY{c+c1}{\PYZsh{} 8}
\PY{p}{)}
\end{Verbatim}
\end{tcolorbox}

    First, we will make a plot only for the cities:

    \begin{tcolorbox}[breakable, size=fbox, boxrule=1pt, pad at break*=1mm,colback=cellbackground, colframe=cellborder]
\prompt{In}{incolor}{10}{\boxspacing}
\begin{Verbatim}[commandchars=\\\{\}]
\PY{k+kn}{import} \PY{n+nn}{matplotlib.pyplot} \PY{k+kn}{as} \PY{n+nn}{plt}
\PY{n}{x\PYZus{}coords} \PY{o}{=} \PY{p}{[}\PY{n}{c}\PY{p}{[}\PY{l+m+mi}{0}\PY{p}{]} \PY{k}{for} \PY{n}{c} \PY{o+ow}{in} \PY{n}{coords}\PY{p}{]}
\PY{n}{y\PYZus{}coords} \PY{o}{=} \PY{p}{[}\PY{n}{c}\PY{p}{[}\PY{l+m+mi}{1}\PY{p}{]} \PY{k}{for} \PY{n}{c} \PY{o+ow}{in} \PY{n}{coords}\PY{p}{]}
\PY{o}{\PYZpc{}}\PY{n}{matplotlib} \PY{n}{inline}
\PY{n}{plt}\PY{o}{.}\PY{n}{scatter}\PY{p}{(}\PY{n}{x\PYZus{}coords}\PY{p}{,} \PY{n}{y\PYZus{}coords}\PY{p}{)}
\PY{k}{for} \PY{n}{city}\PY{p}{,} \PY{n}{tag} \PY{o+ow}{in} \PY{n+nb}{enumerate}\PY{p}{(}\PY{n}{cities}\PY{p}{)}\PY{p}{:}
    \PY{n}{plt}\PY{o}{.}\PY{n}{annotate}\PY{p}{(}\PY{n}{tag}\PY{p}{,} \PY{p}{(}\PY{n}{x\PYZus{}coords}\PY{p}{[}\PY{n}{city}\PY{p}{]}\PY{p}{,} \PY{n}{y\PYZus{}coords}\PY{p}{[}\PY{n}{city}\PY{p}{]}\PY{p}{)}\PY{p}{)}
\PY{n}{plt}\PY{o}{.}\PY{n}{show}\PY{p}{(}\PY{p}{)}
\end{Verbatim}
\end{tcolorbox}

    \begin{center}
    \adjustimage{max size={0.9\linewidth}{0.9\paperheight}}{homework3_files/homework3_22_0.pdf}
    \end{center}
    { \hspace*{\fill} \\}
    
    Now we can declare the function for the loop:

    \begin{tcolorbox}[breakable, size=fbox, boxrule=1pt, pad at break*=1mm,colback=cellbackground, colframe=cellborder]
\prompt{In}{incolor}{11}{\boxspacing}
\begin{Verbatim}[commandchars=\\\{\}]
\PY{k}{def} \PY{n+nf}{plot\PYZus{}tour}\PY{p}{(}\PY{p}{)}\PY{p}{:}
    \PY{l+s+sd}{\PYZsq{}\PYZsq{}\PYZsq{}Plots the path of the subtour.\PYZsq{}\PYZsq{}\PYZsq{}}
    \PY{n}{plt}\PY{o}{.}\PY{n}{scatter}\PY{p}{(}\PY{n}{x\PYZus{}coords}\PY{p}{,} \PY{n}{y\PYZus{}coords}\PY{p}{)}
    \PY{k}{for} \PY{n}{city}\PY{p}{,} \PY{n}{tag} \PY{o+ow}{in} \PY{n+nb}{enumerate}\PY{p}{(}\PY{n}{cities}\PY{p}{)}\PY{p}{:}
        \PY{n}{plt}\PY{o}{.}\PY{n}{annotate}\PY{p}{(}\PY{n}{tag}\PY{p}{,} \PY{p}{(}\PY{n}{x\PYZus{}coords}\PY{p}{[}\PY{n}{city}\PY{p}{]}\PY{p}{,} \PY{n}{y\PYZus{}coords}\PY{p}{[}\PY{n}{city}\PY{p}{]}\PY{p}{)}\PY{p}{)}
    \PY{k}{global} \PY{n}{tour}
    \PY{n}{sorted\PYZus{}coords} \PY{o}{=} \PY{p}{[}\PY{n}{coords}\PY{p}{[}\PY{n}{k}\PY{p}{]} \PY{k}{for} \PY{n}{k} \PY{o+ow}{in} \PY{p}{[}\PY{n}{t} \PY{o}{\PYZhy{}} \PY{l+m+mi}{1} \PY{k}{for} \PY{n}{t} \PY{o+ow}{in} \PY{n}{tour}\PY{p}{]}\PY{p}{]}
    \PY{n}{x\PYZus{}values} \PY{o}{=} \PY{p}{[}\PY{n}{c}\PY{p}{[}\PY{l+m+mi}{0}\PY{p}{]} \PY{k}{for} \PY{n}{c} \PY{o+ow}{in} \PY{n}{sorted\PYZus{}coords}\PY{p}{]}
    \PY{n}{y\PYZus{}values} \PY{o}{=} \PY{p}{[}\PY{n}{c}\PY{p}{[}\PY{l+m+mi}{1}\PY{p}{]} \PY{k}{for} \PY{n}{c} \PY{o+ow}{in} \PY{n}{sorted\PYZus{}coords}\PY{p}{]}
    \PY{n}{plt}\PY{o}{.}\PY{n}{plot}\PY{p}{(}\PY{n}{x\PYZus{}values}\PY{p}{,} \PY{n}{y\PYZus{}values}\PY{p}{)}
    \PY{n}{plt}\PY{o}{.}\PY{n}{show}\PY{p}{(}\PY{p}{)}
\end{Verbatim}
\end{tcolorbox}

    And plot the path of the current subtour \(T\):

    \begin{tcolorbox}[breakable, size=fbox, boxrule=1pt, pad at break*=1mm,colback=cellbackground, colframe=cellborder]
\prompt{In}{incolor}{12}{\boxspacing}
\begin{Verbatim}[commandchars=\\\{\}]
\PY{n}{plot\PYZus{}tour}\PY{p}{(}\PY{p}{)}
\end{Verbatim}
\end{tcolorbox}

    \begin{center}
    \adjustimage{max size={0.9\linewidth}{0.9\paperheight}}{homework3_files/homework3_26_0.pdf}
    \end{center}
    { \hspace*{\fill} \\}
    
    \hypertarget{nearest-insertion-heuristic}{%
\subsection{Nearest Insertion
Heuristic}\label{nearest-insertion-heuristic}}

In order to solve the TSP, we will to apply the \textbf{nearest
insertion heuristic}.

This technique consists of selecting the closest city to the current
tour, which is actually a subtour, because we haven't visited all cities
yet. Then we choose an edge of the subtour in which we will insert the
selected city such that the cost of breaking it is minimal.

    \hypertarget{distance-function}{%
\subsubsection{Distance function}\label{distance-function}}

The closest city is the one whose sum of distances to every visited city
in the subtour is the smallest.

    \[
\newcommand{\dist}{\mathop{\mathrm{dist}}}
\dist(c \in \bar{C}) = \sum_{t \in T} d_{ct}
\]

    \begin{tcolorbox}[breakable, size=fbox, boxrule=1pt, pad at break*=1mm,colback=cellbackground, colframe=cellborder]
\prompt{In}{incolor}{13}{\boxspacing}
\begin{Verbatim}[commandchars=\\\{\}]
\PY{k}{def} \PY{n+nf}{dist}\PY{p}{(}\PY{n}{candidate\PYZus{}city}\PY{p}{)}\PY{p}{:}
    \PY{l+s+sd}{\PYZsq{}\PYZsq{}\PYZsq{}Calculates the distance between candidate\PYZus{}city and the current subtour.\PYZsq{}\PYZsq{}\PYZsq{}}
    \PY{k}{return} \PY{n+nb}{sum}\PY{p}{(}\PY{n}{distances}\PY{p}{[}\PY{n}{candidate\PYZus{}city} \PY{o}{\PYZhy{}} \PY{l+m+mi}{1}\PY{p}{]}\PY{p}{[}\PY{n}{t} \PY{o}{\PYZhy{}} \PY{l+m+mi}{1}\PY{p}{]} \PY{k}{for} \PY{n}{t} \PY{o+ow}{in} \PY{n}{tour}\PY{p}{)}
\end{Verbatim}
\end{tcolorbox}

    \hypertarget{cost-function}{%
\subsubsection{Cost function}\label{cost-function}}

Having the next city to visit already selected, we proceed to choose an
edge of the subtour in which the selected city will be inserted. This is
determined by calculating the cost of removing the existing edge and
creating two new ones, in order to connect the selected city \(c_s\),
and choose the smallest cost.

    \[
\newcommand{\cost}{\mathop{\mathrm{cost}}}
\cost(c_s, E) = d_{c_s e_0} + d_{c_s e_1} - d_{e_0 e_1}
\] \[
e \in E
\] \[
\forall E \subset T, |E| = 2
\]

    \(E\) represents an edge of the tour \(T\), which has only the two
cities that form that edge, being \(e_0\) and \(e_1\). This two cities
also belong to \(T\).

    \begin{tcolorbox}[breakable, size=fbox, boxrule=1pt, pad at break*=1mm,colback=cellbackground, colframe=cellborder]
\prompt{In}{incolor}{14}{\boxspacing}
\begin{Verbatim}[commandchars=\\\{\}]
\PY{k}{def} \PY{n+nf}{cost}\PY{p}{(}\PY{n}{selected\PYZus{}city}\PY{p}{,} \PY{n}{edge}\PY{p}{)}\PY{p}{:}
    \PY{l+s+sd}{\PYZsq{}\PYZsq{}\PYZsq{}Calculates the cost of inserting the selected city in the edge.\PYZsq{}\PYZsq{}\PYZsq{}}
    \PY{n}{e0}\PY{p}{,} \PY{n}{e1} \PY{o}{=} \PY{n}{edge}
    \PY{n}{distance\PYZus{}cs\PYZus{}e0} \PY{o}{=} \PY{n}{distances}\PY{p}{[}\PY{n}{selected\PYZus{}city} \PY{o}{\PYZhy{}} \PY{l+m+mi}{1}\PY{p}{]}\PY{p}{[}\PY{n}{e0} \PY{o}{\PYZhy{}} \PY{l+m+mi}{1}\PY{p}{]}
    \PY{n}{distance\PYZus{}cs\PYZus{}e1} \PY{o}{=} \PY{n}{distances}\PY{p}{[}\PY{n}{selected\PYZus{}city} \PY{o}{\PYZhy{}} \PY{l+m+mi}{1}\PY{p}{]}\PY{p}{[}\PY{n}{e1} \PY{o}{\PYZhy{}} \PY{l+m+mi}{1}\PY{p}{]}
    \PY{n}{distance\PYZus{}e0\PYZus{}e1} \PY{o}{=} \PY{n}{distances}\PY{p}{[}\PY{n}{e0} \PY{o}{\PYZhy{}} \PY{l+m+mi}{1}\PY{p}{]}\PY{p}{[}\PY{n}{e1} \PY{o}{\PYZhy{}} \PY{l+m+mi}{1}\PY{p}{]}
    \PY{k}{return} \PY{n}{distance\PYZus{}cs\PYZus{}e0} \PY{o}{+} \PY{n}{distance\PYZus{}cs\PYZus{}e1} \PY{o}{\PYZhy{}} \PY{n}{distance\PYZus{}e0\PYZus{}e1}
\end{Verbatim}
\end{tcolorbox}

    \hypertarget{loop}{%
\subsection{Loop}\label{loop}}

Now we will enter in a loop to check which remaining cities,
\(\forall c \in \overline{C}\), to visit next and add them in an order
to our tour \(T\).

The body of the whole loop looks as follows:

    \textbf{ while \(\bar{C} \ne \emptyset\) do\\
\(\hbox{}\qquad c_s \leftarrow \newcommand{\argmin}{\mathop{\mathrm{argmin}}\limits} \argmin_{c \in \bar{C}} \newcommand{\dist}{\mathop{\mathrm{dist}}} \{\dist (c)\}\)\\
\(\hbox{}\qquad k \leftarrow \newcommand{\argmin}{\mathop{\mathrm{argmin}}\limits} \argmin_{E \subset T} \newcommand{\cost}{\mathop{\mathrm{cost}}} \{\cost (c_s, E)\}\)\\
\(\hbox{}\qquad T (t_1, \ldots, t_n) \leftarrow T (t_1, \ldots, t_k, c_s, t_{k+1}, \ldots, t_n)\)\\
\(\hbox{}\qquad \bar{C} \leftarrow \bar{C} \setminus \{c_s\}\)\\
end while }

    To calculate
\(c_s \leftarrow \newcommand{\argmin}{\mathop{\mathrm{argmin}}\limits} \argmin_{c \in \bar{C}} \newcommand{\dist}{\mathop{\mathrm{dist}}} \{\dist (c)\}\),
we will use the following function which will help us getting the actual
nearest city:

    \begin{tcolorbox}[breakable, size=fbox, boxrule=1pt, pad at break*=1mm,colback=cellbackground, colframe=cellborder]
\prompt{In}{incolor}{15}{\boxspacing}
\begin{Verbatim}[commandchars=\\\{\}]
\PY{k}{def} \PY{n+nf}{nearest}\PY{p}{(}\PY{p}{)}\PY{p}{:}
    \PY{l+s+sd}{\PYZsq{}\PYZsq{}\PYZsq{}Returns a list of distances between each city and the subtour.\PYZsq{}\PYZsq{}\PYZsq{}}
    \PY{k}{return} \PY{p}{[}\PY{p}{(}\PY{n}{city}\PY{p}{,} \PY{n}{dist}\PY{p}{(}\PY{n}{city}\PY{p}{)}\PY{p}{)} \PY{k}{for} \PY{n}{\PYZus{}}\PY{p}{,} \PY{n}{city} \PY{o+ow}{in} \PY{n+nb}{enumerate}\PY{p}{(}\PY{n}{cities\PYZus{}bar}\PY{p}{)}\PY{p}{]}
\end{Verbatim}
\end{tcolorbox}

    In the same way, to calculate
\(k \leftarrow \newcommand{\argmin}{\mathop{\mathrm{argmin}}\limits} \argmin_{E \subset T} \newcommand{\cost}{\mathop{\mathrm{cost}}} \{\cost (c_s, E)\}\),
we will use the following function:

    \begin{tcolorbox}[breakable, size=fbox, boxrule=1pt, pad at break*=1mm,colback=cellbackground, colframe=cellborder]
\prompt{In}{incolor}{16}{\boxspacing}
\begin{Verbatim}[commandchars=\\\{\}]
\PY{k}{def} \PY{n+nf}{argmin\PYZus{}cost}\PY{p}{(}\PY{n}{city}\PY{p}{)}\PY{p}{:}
    \PY{l+s+sd}{\PYZsq{}\PYZsq{}\PYZsq{}Returns a list of costs of each city in the subtour.\PYZsq{}\PYZsq{}\PYZsq{}}
    \PY{n}{indexed\PYZus{}costs} \PY{o}{=} \PY{n+nb}{list}\PY{p}{(}\PY{p}{)}
    \PY{k}{for} \PY{n}{i} \PY{o+ow}{in} \PY{n+nb}{range}\PY{p}{(}\PY{n+nb}{len}\PY{p}{(}\PY{n}{tour}\PY{p}{)} \PY{o}{\PYZhy{}} \PY{l+m+mi}{1}\PY{p}{)}\PY{p}{:}
        \PY{n}{edge} \PY{o}{=} \PY{p}{(}\PY{n}{tour}\PY{p}{[}\PY{n}{i}\PY{p}{]}\PY{p}{,} \PY{n}{tour}\PY{p}{[}\PY{n}{i} \PY{o}{+} \PY{l+m+mi}{1}\PY{p}{]}\PY{p}{)}
        \PY{n}{indexed\PYZus{}costs}\PY{o}{.}\PY{n}{append}\PY{p}{(}\PY{p}{(}\PY{n}{i} \PY{o}{+} \PY{l+m+mi}{1}\PY{p}{,} \PY{n}{cost}\PY{p}{(}\PY{n}{city}\PY{p}{,} \PY{n}{edge}\PY{p}{)}\PY{p}{)}\PY{p}{)}
    \PY{k}{return} \PY{n}{indexed\PYZus{}costs}
\end{Verbatim}
\end{tcolorbox}

    And finally, to update the tour we will use a function to insert the
selected city \(c_s\), obtained by checking the results of
\texttt{nearest()}, next to the city \(t_k\) in the subtour that has the
smallest cost, obtained by checking the results of
\texttt{argmin\_cost()}, as follows:

    \begin{tcolorbox}[breakable, size=fbox, boxrule=1pt, pad at break*=1mm,colback=cellbackground, colframe=cellborder]
\prompt{In}{incolor}{17}{\boxspacing}
\begin{Verbatim}[commandchars=\\\{\}]
\PY{k}{def} \PY{n+nf}{update\PYZus{}tour}\PY{p}{(}\PY{n}{city}\PY{p}{,} \PY{n}{k}\PY{p}{)}\PY{p}{:}
    \PY{l+s+sd}{\PYZsq{}\PYZsq{}\PYZsq{}Inserts the selected city in the k index of the tour.\PYZsq{}\PYZsq{}\PYZsq{}}
    \PY{k}{global} \PY{n}{tour}
    \PY{n}{tour} \PY{o}{=} \PY{n}{tour}\PY{p}{[}\PY{p}{:}\PY{n}{k}\PY{p}{]} \PY{o}{+} \PY{p}{[}\PY{n}{city}\PY{p}{]} \PY{o}{+} \PY{n}{tour}\PY{p}{[}\PY{n}{k}\PY{p}{:}\PY{p}{]}
    \PY{k}{return} \PY{n}{tour}
\end{Verbatim}
\end{tcolorbox}

    \hypertarget{first-iteration}{%
\subsubsection{First iteration}\label{first-iteration}}

As we already know that we need to visit six more cities to complete the
tour, we can start by checking which unvisited city is the closest from
our subtour:

    \begin{tcolorbox}[breakable, size=fbox, boxrule=1pt, pad at break*=1mm,colback=cellbackground, colframe=cellborder]
\prompt{In}{incolor}{18}{\boxspacing}
\begin{Verbatim}[commandchars=\\\{\}]
\PY{n}{nearest}\PY{p}{(}\PY{p}{)}
\end{Verbatim}
\end{tcolorbox}

            \begin{tcolorbox}[breakable, size=fbox, boxrule=.5pt, pad at break*=1mm, opacityfill=0]
\prompt{Out}{outcolor}{18}{\boxspacing}
\begin{Verbatim}[commandchars=\\\{\}]
[(1, 172), (2, 151), (3, 207), (4, 156), (5, 73), (6, 184)]
\end{Verbatim}
\end{tcolorbox}
        
    We can see that the city \(5\) is the \emph{nearest} to the subtour,
\(\therefore\) \(c_s = 5\).

Now we want to know the index of the city in the subtour that will be
previous to \(5\). In this case, because we are in the first iteration,
there's only one edge available, so we can insert \(5\) between \(7\)
and \(8\) or append it before the end; in both cases, the total length
of the subtour would be the same because the cost is the same. Let's
append it right after \(8\):

    \begin{tcolorbox}[breakable, size=fbox, boxrule=1pt, pad at break*=1mm,colback=cellbackground, colframe=cellborder]
\prompt{In}{incolor}{19}{\boxspacing}
\begin{Verbatim}[commandchars=\\\{\}]
\PY{n}{update\PYZus{}tour}\PY{p}{(}\PY{l+m+mi}{5}\PY{p}{,} \PY{l+m+mi}{2}\PY{p}{)}
\end{Verbatim}
\end{tcolorbox}

            \begin{tcolorbox}[breakable, size=fbox, boxrule=.5pt, pad at break*=1mm, opacityfill=0]
\prompt{Out}{outcolor}{19}{\boxspacing}
\begin{Verbatim}[commandchars=\\\{\}]
[7, 8, 5, 7]
\end{Verbatim}
\end{tcolorbox}
        
    Let's see how the path looks:

    \begin{tcolorbox}[breakable, size=fbox, boxrule=1pt, pad at break*=1mm,colback=cellbackground, colframe=cellborder]
\prompt{In}{incolor}{20}{\boxspacing}
\begin{Verbatim}[commandchars=\\\{\}]
\PY{n}{plot\PYZus{}tour}\PY{p}{(}\PY{p}{)}
\end{Verbatim}
\end{tcolorbox}

    \begin{center}
    \adjustimage{max size={0.9\linewidth}{0.9\paperheight}}{homework3_files/homework3_48_0.pdf}
    \end{center}
    { \hspace*{\fill} \\}
    
    \(\bar{C}\) needs to be updated as the subtour \(T\) grows:

\(\bar{C} \leftarrow \bar{C} \setminus \{5\}\)

    \begin{tcolorbox}[breakable, size=fbox, boxrule=1pt, pad at break*=1mm,colback=cellbackground, colframe=cellborder]
\prompt{In}{incolor}{21}{\boxspacing}
\begin{Verbatim}[commandchars=\\\{\}]
\PY{n}{cities\PYZus{}bar}\PY{o}{.}\PY{n}{remove}\PY{p}{(}\PY{l+m+mi}{5}\PY{p}{)}
\PY{n}{cities\PYZus{}bar}
\end{Verbatim}
\end{tcolorbox}

            \begin{tcolorbox}[breakable, size=fbox, boxrule=.5pt, pad at break*=1mm, opacityfill=0]
\prompt{Out}{outcolor}{21}{\boxspacing}
\begin{Verbatim}[commandchars=\\\{\}]
[1, 2, 3, 4, 6]
\end{Verbatim}
\end{tcolorbox}
        
    \hypertarget{second-iteration}{%
\subsubsection{Second iteration}\label{second-iteration}}

Let's choose the nearest city to our subtour \(T\):

    \begin{tcolorbox}[breakable, size=fbox, boxrule=1pt, pad at break*=1mm,colback=cellbackground, colframe=cellborder]
\prompt{In}{incolor}{22}{\boxspacing}
\begin{Verbatim}[commandchars=\\\{\}]
\PY{n}{nearest}\PY{p}{(}\PY{p}{)}
\end{Verbatim}
\end{tcolorbox}

            \begin{tcolorbox}[breakable, size=fbox, boxrule=.5pt, pad at break*=1mm, opacityfill=0]
\prompt{Out}{outcolor}{22}{\boxspacing}
\begin{Verbatim}[commandchars=\\\{\}]
[(1, 205), (2, 216), (3, 282), (4, 196), (6, 246)]
\end{Verbatim}
\end{tcolorbox}
        
    The city \(4\) is the closest one, with a total distance of \(196\). Now
let's find the best edge to break from \(T\):

    \begin{tcolorbox}[breakable, size=fbox, boxrule=1pt, pad at break*=1mm,colback=cellbackground, colframe=cellborder]
\prompt{In}{incolor}{23}{\boxspacing}
\begin{Verbatim}[commandchars=\\\{\}]
\PY{n}{argmin\PYZus{}cost}\PY{p}{(}\PY{l+m+mi}{4}\PY{p}{)}
\end{Verbatim}
\end{tcolorbox}

            \begin{tcolorbox}[breakable, size=fbox, boxrule=.5pt, pad at break*=1mm, opacityfill=0]
\prompt{Out}{outcolor}{23}{\boxspacing}
\begin{Verbatim}[commandchars=\\\{\}]
[(1, 90), (2, 75), (3, 64)]
\end{Verbatim}
\end{tcolorbox}
        
    The edge between cities \(t_3\) and \(t_4\) is the cheapest one to
break. These cities are \(5\) and \(7\), which means that the selected
edge is the last one of the subtour:

    \begin{tcolorbox}[breakable, size=fbox, boxrule=1pt, pad at break*=1mm,colback=cellbackground, colframe=cellborder]
\prompt{In}{incolor}{24}{\boxspacing}
\begin{Verbatim}[commandchars=\\\{\}]
\PY{n}{update\PYZus{}tour}\PY{p}{(}\PY{l+m+mi}{4}\PY{p}{,} \PY{l+m+mi}{3}\PY{p}{)}
\end{Verbatim}
\end{tcolorbox}

            \begin{tcolorbox}[breakable, size=fbox, boxrule=.5pt, pad at break*=1mm, opacityfill=0]
\prompt{Out}{outcolor}{24}{\boxspacing}
\begin{Verbatim}[commandchars=\\\{\}]
[7, 8, 5, 4, 7]
\end{Verbatim}
\end{tcolorbox}
        
    And how is the path going?

    \begin{tcolorbox}[breakable, size=fbox, boxrule=1pt, pad at break*=1mm,colback=cellbackground, colframe=cellborder]
\prompt{In}{incolor}{25}{\boxspacing}
\begin{Verbatim}[commandchars=\\\{\}]
\PY{n}{plot\PYZus{}tour}\PY{p}{(}\PY{p}{)}
\end{Verbatim}
\end{tcolorbox}

    \begin{center}
    \adjustimage{max size={0.9\linewidth}{0.9\paperheight}}{homework3_files/homework3_58_0.pdf}
    \end{center}
    { \hspace*{\fill} \\}
    
    Now we must update \(\bar{C}\) by removing the city \(4\):

    \begin{tcolorbox}[breakable, size=fbox, boxrule=1pt, pad at break*=1mm,colback=cellbackground, colframe=cellborder]
\prompt{In}{incolor}{26}{\boxspacing}
\begin{Verbatim}[commandchars=\\\{\}]
\PY{n}{cities\PYZus{}bar}\PY{o}{.}\PY{n}{remove}\PY{p}{(}\PY{l+m+mi}{4}\PY{p}{)}
\PY{n}{cities\PYZus{}bar}
\end{Verbatim}
\end{tcolorbox}

            \begin{tcolorbox}[breakable, size=fbox, boxrule=.5pt, pad at break*=1mm, opacityfill=0]
\prompt{Out}{outcolor}{26}{\boxspacing}
\begin{Verbatim}[commandchars=\\\{\}]
[1, 2, 3, 6]
\end{Verbatim}
\end{tcolorbox}
        
    \hypertarget{third-iteration}{%
\subsubsection{Third iteration}\label{third-iteration}}

We want to select the closest unvisited city to our subtour \(T\):

    \begin{tcolorbox}[breakable, size=fbox, boxrule=1pt, pad at break*=1mm,colback=cellbackground, colframe=cellborder]
\prompt{In}{incolor}{27}{\boxspacing}
\begin{Verbatim}[commandchars=\\\{\}]
\PY{n}{nearest}\PY{p}{(}\PY{p}{)}
\end{Verbatim}
\end{tcolorbox}

            \begin{tcolorbox}[breakable, size=fbox, boxrule=.5pt, pad at break*=1mm, opacityfill=0]
\prompt{Out}{outcolor}{27}{\boxspacing}
\begin{Verbatim}[commandchars=\\\{\}]
[(1, 243), (2, 275), (3, 332), (6, 279)]
\end{Verbatim}
\end{tcolorbox}
        
    City \(1\) is the nearest to \(T\). Now to find where to insert it:

    \begin{tcolorbox}[breakable, size=fbox, boxrule=1pt, pad at break*=1mm,colback=cellbackground, colframe=cellborder]
\prompt{In}{incolor}{28}{\boxspacing}
\begin{Verbatim}[commandchars=\\\{\}]
\PY{n}{argmin\PYZus{}cost}\PY{p}{(}\PY{l+m+mi}{1}\PY{p}{)}
\end{Verbatim}
\end{tcolorbox}

            \begin{tcolorbox}[breakable, size=fbox, boxrule=.5pt, pad at break*=1mm, opacityfill=0]
\prompt{Out}{outcolor}{28}{\boxspacing}
\begin{Verbatim}[commandchars=\\\{\}]
[(1, 97), (2, 66), (3, 31), (4, 47)]
\end{Verbatim}
\end{tcolorbox}
        
    In order to minimize the cost of inserting the selected city, we will
choose the \(t_3\) to be the previous city to the selected one:

    \begin{tcolorbox}[breakable, size=fbox, boxrule=1pt, pad at break*=1mm,colback=cellbackground, colframe=cellborder]
\prompt{In}{incolor}{29}{\boxspacing}
\begin{Verbatim}[commandchars=\\\{\}]
\PY{n}{update\PYZus{}tour}\PY{p}{(}\PY{l+m+mi}{1}\PY{p}{,} \PY{l+m+mi}{3}\PY{p}{)}
\end{Verbatim}
\end{tcolorbox}

            \begin{tcolorbox}[breakable, size=fbox, boxrule=.5pt, pad at break*=1mm, opacityfill=0]
\prompt{Out}{outcolor}{29}{\boxspacing}
\begin{Verbatim}[commandchars=\\\{\}]
[7, 8, 5, 1, 4, 7]
\end{Verbatim}
\end{tcolorbox}
        
    The path of the subtour looks as follows:

    \begin{tcolorbox}[breakable, size=fbox, boxrule=1pt, pad at break*=1mm,colback=cellbackground, colframe=cellborder]
\prompt{In}{incolor}{30}{\boxspacing}
\begin{Verbatim}[commandchars=\\\{\}]
\PY{n}{plot\PYZus{}tour}\PY{p}{(}\PY{p}{)}
\end{Verbatim}
\end{tcolorbox}

    \begin{center}
    \adjustimage{max size={0.9\linewidth}{0.9\paperheight}}{homework3_files/homework3_68_0.pdf}
    \end{center}
    { \hspace*{\fill} \\}
    
    To finish the iteration, we will remove city \(1\) from \(\bar{C}\):

    \begin{tcolorbox}[breakable, size=fbox, boxrule=1pt, pad at break*=1mm,colback=cellbackground, colframe=cellborder]
\prompt{In}{incolor}{31}{\boxspacing}
\begin{Verbatim}[commandchars=\\\{\}]
\PY{n}{cities\PYZus{}bar}\PY{o}{.}\PY{n}{remove}\PY{p}{(}\PY{l+m+mi}{1}\PY{p}{)}
\PY{n}{cities\PYZus{}bar}
\end{Verbatim}
\end{tcolorbox}

            \begin{tcolorbox}[breakable, size=fbox, boxrule=.5pt, pad at break*=1mm, opacityfill=0]
\prompt{Out}{outcolor}{31}{\boxspacing}
\begin{Verbatim}[commandchars=\\\{\}]
[2, 3, 6]
\end{Verbatim}
\end{tcolorbox}
        
    \hypertarget{fourth-iteration}{%
\subsubsection{Fourth iteration}\label{fourth-iteration}}

Let's calculate the distance from each unvisited city to our subtour
\(T\):

    \begin{tcolorbox}[breakable, size=fbox, boxrule=1pt, pad at break*=1mm,colback=cellbackground, colframe=cellborder]
\prompt{In}{incolor}{32}{\boxspacing}
\begin{Verbatim}[commandchars=\\\{\}]
\PY{n}{nearest}\PY{p}{(}\PY{p}{)}
\end{Verbatim}
\end{tcolorbox}

            \begin{tcolorbox}[breakable, size=fbox, boxrule=.5pt, pad at break*=1mm, opacityfill=0]
\prompt{Out}{outcolor}{32}{\boxspacing}
\begin{Verbatim}[commandchars=\\\{\}]
[(2, 364), (3, 419), (6, 350)]
\end{Verbatim}
\end{tcolorbox}
        
    We will visit city \(6\) next because it's the nearest to \(T\).
However, we need to know after what city in the subtour we will visit
\(c_s = 6\):

    \begin{tcolorbox}[breakable, size=fbox, boxrule=1pt, pad at break*=1mm,colback=cellbackground, colframe=cellborder]
\prompt{In}{incolor}{33}{\boxspacing}
\begin{Verbatim}[commandchars=\\\{\}]
\PY{n}{argmin\PYZus{}cost}\PY{p}{(}\PY{l+m+mi}{6}\PY{p}{)}
\end{Verbatim}
\end{tcolorbox}

            \begin{tcolorbox}[breakable, size=fbox, boxrule=.5pt, pad at break*=1mm, opacityfill=0]
\prompt{Out}{outcolor}{33}{\boxspacing}
\begin{Verbatim}[commandchars=\\\{\}]
[(1, 111), (2, 111), (3, 100), (4, 66), (5, 40)]
\end{Verbatim}
\end{tcolorbox}
        
    Let's insert city \(6\) right after \(t_5\), which is city \(4\):

    \begin{tcolorbox}[breakable, size=fbox, boxrule=1pt, pad at break*=1mm,colback=cellbackground, colframe=cellborder]
\prompt{In}{incolor}{34}{\boxspacing}
\begin{Verbatim}[commandchars=\\\{\}]
\PY{n}{update\PYZus{}tour}\PY{p}{(}\PY{l+m+mi}{6}\PY{p}{,} \PY{l+m+mi}{5}\PY{p}{)}
\end{Verbatim}
\end{tcolorbox}

            \begin{tcolorbox}[breakable, size=fbox, boxrule=.5pt, pad at break*=1mm, opacityfill=0]
\prompt{Out}{outcolor}{34}{\boxspacing}
\begin{Verbatim}[commandchars=\\\{\}]
[7, 8, 5, 1, 4, 6, 7]
\end{Verbatim}
\end{tcolorbox}
        
    \begin{tcolorbox}[breakable, size=fbox, boxrule=1pt, pad at break*=1mm,colback=cellbackground, colframe=cellborder]
\prompt{In}{incolor}{35}{\boxspacing}
\begin{Verbatim}[commandchars=\\\{\}]
\PY{n}{plot\PYZus{}tour}\PY{p}{(}\PY{p}{)}
\end{Verbatim}
\end{tcolorbox}

    \begin{center}
    \adjustimage{max size={0.9\linewidth}{0.9\paperheight}}{homework3_files/homework3_77_0.pdf}
    \end{center}
    { \hspace*{\fill} \\}
    
    And remove it from \(\bar{C}\):

    \begin{tcolorbox}[breakable, size=fbox, boxrule=1pt, pad at break*=1mm,colback=cellbackground, colframe=cellborder]
\prompt{In}{incolor}{36}{\boxspacing}
\begin{Verbatim}[commandchars=\\\{\}]
\PY{n}{cities\PYZus{}bar}\PY{o}{.}\PY{n}{remove}\PY{p}{(}\PY{l+m+mi}{6}\PY{p}{)}
\PY{n}{cities\PYZus{}bar}
\end{Verbatim}
\end{tcolorbox}

            \begin{tcolorbox}[breakable, size=fbox, boxrule=.5pt, pad at break*=1mm, opacityfill=0]
\prompt{Out}{outcolor}{36}{\boxspacing}
\begin{Verbatim}[commandchars=\\\{\}]
[2, 3]
\end{Verbatim}
\end{tcolorbox}
        
    \hypertarget{fifth-iteration}{%
\subsubsection{Fifth iteration}\label{fifth-iteration}}

This is the last iteration that requires to check which city to visit
next, because there are only two left.

    \begin{tcolorbox}[breakable, size=fbox, boxrule=1pt, pad at break*=1mm,colback=cellbackground, colframe=cellborder]
\prompt{In}{incolor}{37}{\boxspacing}
\begin{Verbatim}[commandchars=\\\{\}]
\PY{n}{nearest}\PY{p}{(}\PY{p}{)}
\end{Verbatim}
\end{tcolorbox}

            \begin{tcolorbox}[breakable, size=fbox, boxrule=.5pt, pad at break*=1mm, opacityfill=0]
\prompt{Out}{outcolor}{37}{\boxspacing}
\begin{Verbatim}[commandchars=\\\{\}]
[(2, 403), (3, 436)]
\end{Verbatim}
\end{tcolorbox}
        
    City \(2\) is the next to be visited, so let's find the cheapest edge of
the subtour \(T\) to break:

    \begin{tcolorbox}[breakable, size=fbox, boxrule=1pt, pad at break*=1mm,colback=cellbackground, colframe=cellborder]
\prompt{In}{incolor}{38}{\boxspacing}
\begin{Verbatim}[commandchars=\\\{\}]
\PY{n}{argmin\PYZus{}cost}\PY{p}{(}\PY{l+m+mi}{2}\PY{p}{)}
\end{Verbatim}
\end{tcolorbox}

            \begin{tcolorbox}[breakable, size=fbox, boxrule=.5pt, pad at break*=1mm, opacityfill=0]
\prompt{Out}{outcolor}{38}{\boxspacing}
\begin{Verbatim}[commandchars=\\\{\}]
[(1, 90), (2, 105), (3, 121), (4, 110), (5, 65), (6, 27)]
\end{Verbatim}
\end{tcolorbox}
        
    Again, we are going to break the last edge of the current subtour:

    \begin{tcolorbox}[breakable, size=fbox, boxrule=1pt, pad at break*=1mm,colback=cellbackground, colframe=cellborder]
\prompt{In}{incolor}{39}{\boxspacing}
\begin{Verbatim}[commandchars=\\\{\}]
\PY{n}{update\PYZus{}tour}\PY{p}{(}\PY{l+m+mi}{2}\PY{p}{,} \PY{l+m+mi}{6}\PY{p}{)}
\end{Verbatim}
\end{tcolorbox}

            \begin{tcolorbox}[breakable, size=fbox, boxrule=.5pt, pad at break*=1mm, opacityfill=0]
\prompt{Out}{outcolor}{39}{\boxspacing}
\begin{Verbatim}[commandchars=\\\{\}]
[7, 8, 5, 1, 4, 6, 2, 7]
\end{Verbatim}
\end{tcolorbox}
        
    Let's take a look at how the path is going:

    \begin{tcolorbox}[breakable, size=fbox, boxrule=1pt, pad at break*=1mm,colback=cellbackground, colframe=cellborder]
\prompt{In}{incolor}{40}{\boxspacing}
\begin{Verbatim}[commandchars=\\\{\}]
\PY{n}{plot\PYZus{}tour}\PY{p}{(}\PY{p}{)}
\end{Verbatim}
\end{tcolorbox}

    \begin{center}
    \adjustimage{max size={0.9\linewidth}{0.9\paperheight}}{homework3_files/homework3_87_0.pdf}
    \end{center}
    { \hspace*{\fill} \\}
    
    Now remove city \(2\) from \(\bar{C}\):

    \begin{tcolorbox}[breakable, size=fbox, boxrule=1pt, pad at break*=1mm,colback=cellbackground, colframe=cellborder]
\prompt{In}{incolor}{41}{\boxspacing}
\begin{Verbatim}[commandchars=\\\{\}]
\PY{n}{cities\PYZus{}bar}\PY{o}{.}\PY{n}{remove}\PY{p}{(}\PY{l+m+mi}{2}\PY{p}{)}
\PY{n}{cities\PYZus{}bar}
\end{Verbatim}
\end{tcolorbox}

            \begin{tcolorbox}[breakable, size=fbox, boxrule=.5pt, pad at break*=1mm, opacityfill=0]
\prompt{Out}{outcolor}{41}{\boxspacing}
\begin{Verbatim}[commandchars=\\\{\}]
[3]
\end{Verbatim}
\end{tcolorbox}
        
    \hypertarget{sixth-iteration}{%
\subsubsection{Sixth iteration}\label{sixth-iteration}}

We already know that city \(3\) is the last one to visit, so we just
need to get the index of a visited city in our subtour \(T\) whose
breaking cost is the smallest:

    \begin{tcolorbox}[breakable, size=fbox, boxrule=1pt, pad at break*=1mm,colback=cellbackground, colframe=cellborder]
\prompt{In}{incolor}{42}{\boxspacing}
\begin{Verbatim}[commandchars=\\\{\}]
\PY{n}{argmin\PYZus{}cost}\PY{p}{(}\PY{l+m+mi}{3}\PY{p}{)}
\end{Verbatim}
\end{tcolorbox}

            \begin{tcolorbox}[breakable, size=fbox, boxrule=.5pt, pad at break*=1mm, opacityfill=0]
\prompt{Out}{outcolor}{42}{\boxspacing}
\begin{Verbatim}[commandchars=\\\{\}]
[(1, 127), (2, 133), (3, 129), (4, 99), (5, 34), (6, 10), (7, 51)]
\end{Verbatim}
\end{tcolorbox}
        
    Finally, to complete the tour \(T\), we will break the edge that starts
at \(t_6\) and insert city \(3\) there:

    \begin{tcolorbox}[breakable, size=fbox, boxrule=1pt, pad at break*=1mm,colback=cellbackground, colframe=cellborder]
\prompt{In}{incolor}{43}{\boxspacing}
\begin{Verbatim}[commandchars=\\\{\}]
\PY{n}{update\PYZus{}tour}\PY{p}{(}\PY{l+m+mi}{3}\PY{p}{,} \PY{l+m+mi}{6}\PY{p}{)}
\end{Verbatim}
\end{tcolorbox}

            \begin{tcolorbox}[breakable, size=fbox, boxrule=.5pt, pad at break*=1mm, opacityfill=0]
\prompt{Out}{outcolor}{43}{\boxspacing}
\begin{Verbatim}[commandchars=\\\{\}]
[7, 8, 5, 1, 4, 6, 3, 2, 7]
\end{Verbatim}
\end{tcolorbox}
        
    \begin{tcolorbox}[breakable, size=fbox, boxrule=1pt, pad at break*=1mm,colback=cellbackground, colframe=cellborder]
\prompt{In}{incolor}{44}{\boxspacing}
\begin{Verbatim}[commandchars=\\\{\}]
\PY{n}{plot\PYZus{}tour}\PY{p}{(}\PY{p}{)}
\end{Verbatim}
\end{tcolorbox}

    \begin{center}
    \adjustimage{max size={0.9\linewidth}{0.9\paperheight}}{homework3_files/homework3_94_0.pdf}
    \end{center}
    { \hspace*{\fill} \\}
    
    As we can see, the final tour got a circular shape.

By removing the last city left from \(\bar{C}\), we will get an empty
set:

    \begin{tcolorbox}[breakable, size=fbox, boxrule=1pt, pad at break*=1mm,colback=cellbackground, colframe=cellborder]
\prompt{In}{incolor}{45}{\boxspacing}
\begin{Verbatim}[commandchars=\\\{\}]
\PY{n}{cities\PYZus{}bar}\PY{o}{.}\PY{n}{remove}\PY{p}{(}\PY{l+m+mi}{3}\PY{p}{)}
\PY{n}{cities\PYZus{}bar}
\end{Verbatim}
\end{tcolorbox}

            \begin{tcolorbox}[breakable, size=fbox, boxrule=.5pt, pad at break*=1mm, opacityfill=0]
\prompt{Out}{outcolor}{45}{\boxspacing}
\begin{Verbatim}[commandchars=\\\{\}]
[]
\end{Verbatim}
\end{tcolorbox}
        
    This means that our loop has ended, and thus we have calculated a tour
\(T\).

    \hypertarget{conclusion}{%
\section{Conclusion}\label{conclusion}}

We now have our decision variable, the tour \(T\).

\(T \leftarrow \{7, 8, 5, 1, 4, 6, 3, 2, 7\}\)

As we stated at the beginning, the objective function of the TSP is to
minimize the total distance of the tour. Let's calculate this distance:

    \begin{tcolorbox}[breakable, size=fbox, boxrule=1pt, pad at break*=1mm,colback=cellbackground, colframe=cellborder]
\prompt{In}{incolor}{46}{\boxspacing}
\begin{Verbatim}[commandchars=\\\{\}]
\PY{n+nb}{sum}\PY{p}{(}\PY{n}{distances}\PY{p}{[}\PY{n}{tour}\PY{p}{[}\PY{n}{i}\PY{p}{]} \PY{o}{\PYZhy{}} \PY{l+m+mi}{1}\PY{p}{]}\PY{p}{[}\PY{n}{tour}\PY{p}{[}\PY{n}{i} \PY{o}{+} \PY{l+m+mi}{1}\PY{p}{]} \PY{o}{\PYZhy{}} \PY{l+m+mi}{1}\PY{p}{]} \PY{k}{for} \PY{n}{i} \PY{o+ow}{in} \PY{n+nb}{range}\PY{p}{(}\PY{n+nb}{len}\PY{p}{(}\PY{n}{tour}\PY{p}{)} \PY{o}{\PYZhy{}} \PY{l+m+mi}{1}\PY{p}{)}\PY{p}{)}
\end{Verbatim}
\end{tcolorbox}

            \begin{tcolorbox}[breakable, size=fbox, boxrule=.5pt, pad at break*=1mm, opacityfill=0]
\prompt{Out}{outcolor}{46}{\boxspacing}
\begin{Verbatim}[commandchars=\\\{\}]
235
\end{Verbatim}
\end{tcolorbox}
        
    In conclusion, by using the nearest insertion heuristic technique, we
can visit the cities in the order given by \(T\), and when we finish we
will have traveled a total distance of \(235\).


    % Add a bibliography block to the postdoc
    
    
    
\end{document}
